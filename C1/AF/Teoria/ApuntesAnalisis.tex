\documentclass{article}
\usepackage[utf8]{inputenc}
\usepackage{graphicx}
\usepackage[colorlinks]{hyperref}
\usepackage{amsmath}
\usepackage{amssymb}
\usepackage{cancel}
\graphicspath{ {images/} }

\author{Daniel Monjas Miguélez}

\title{Apuntes de Análisis Funcional}

\begin{document}
\maketitle

\newpage 

\tableofcontents

\newpage

\section{Espacios vectoriales de dimensión finita}
En esta sección vamos a dar respuesta a 10 preguntas y vamos a asentar un poco conceptos que se van a usar a lo largo de toda la asignatura:

\begin{enumerate}
\item ¿Qué significa que un espacio vectorial tenga dimensión n?

\item Dado un número natural $n\in \mathbb{N}$, ¿existén espacios vectoriales de dimensión $n$?

\item ¿Qué significa que un espacio tenga dimensión infinita?

\item ¿Existen espacios vectoriales de dimensión infinita?

\item ¿Cuál es el concepto de base de un e.v?

\item ¿Qué relación tiene el concepto de base con la dimensión de un e.v?

\item Dado un espacio vectorial, ¿existe siempre alguna base?

\item Dados 2 e.v sobre un mismo cuerpo $K$ (en adelante supondremos $K=\mathbb{R}$ ó $K=\mathbb{C}$), $V$ y $W$ decimos que $V$ y $W$ son isomorfos entre si existe alguna aplicación $T:V\rightarrow V$, $T$ biyectiva y lineal.

Si la dimensión de $V(K)$ es $n$, es trivial probar que $V$ es isomorfo a $K^n$. Ahora bien, el tema se puede complicar si hablamos de dimensiones infinitas, ya que "no todos los infinitos son iguales", cosa que Cantor dejó claro y por la que se volvió loco. Por ejemplo, si $\chi_0=Car(\mathbb{N})$, $\chi_1=Car(\mathbb{R})$, entonces
\begin{equation*}
\chi_0<\chi_1
\end{equation*}

¿Sabes la respuesta a la siguientes cuestión?
\begin{itemize}
\item ¿$\exists A:\mathbb{N}\subset A\subset \mathbb{R},\:\chi_0<Car(A)<\chi_1?$
\end{itemize}
\textbf{Solución:} No existe, $\chi_1$ es el menor cardinal tal que $\chi_1>\chi_0$. En el siguiente link se da más información \href{https://www.madrimasd.org/blogs/matematicas/tag/aleph-cero}{El paraíso de Cantor}.\\

\textbf{Ejercicio 1}.a) Sea $V(K)$ un e.v de dimensión finita y $L:V\rightarrow V$, lineal. Entonces son equivalentes:
\begin{itemize}
\item $L$ es inyectiva.
\item $L$ es sobreyectiva.
\item $L$ es biyectiva
\end{itemize}

\textbf{Solución:} La solución se va a basar en lo siguiente, sea $f:V\rightarrow V'$ lineal con $dim(V)=n$ y $dim(V')=m$. Entonces
\begin{equation*}
dim(V)=dim(Ker(f))+dim(Im(f))
\end{equation*}

Teniendo en cuenta lo anterior veamos:

$1)\Rightarrow 2)$ Como $L$ es inyectiva, se tiene que $dim(Ker(L))=0$, luego por la fórmula anteriormente enunciada, $dim(V)=0+dim(Im(L))$, luego $dim(Im(L))=dim(V)$ luego $Im(L)=V$, pues $Im(L)\subset V$ y por tanto la aplicación es sobreyectiva.

$2)\Rightarrow 1)$ Ahora como $L$ es sobreyectiva tenemos claramente que $dim(Im(L))=dim(V)$ luego por la fórmula $dim(V)=dim(Ker(L))+dim(V)\Rightarrow dim(Ker(L))=0$, de aquí deducimos que la aplicación es inyectiva.

Por último, como hemos visto que $1)\Leftrightarrow 2)$ se tiene que conque se verifique 1 o 2 tenemos 3, pues si una aplicación es inyectiva y sobreyectiva es biyectiva.\\


b)Sea $L:C([0,1],\mathbb{R})\rightarrow C([0,1],\mathbb{R})$

$\qquad (Lf)(t)=\int_0^t f(s)ds,\quad \left.\begin{array}{c}
\forall f\in C([0,1],\mathbb{R})\\
\forall t\in[0,1] 
\end{array}\right.$

Prueba que $L$ es inyectiva, pero no sobreyectiva.

\textbf{Solución:} Para ver que es inyectiva veamos que si $Lf=0\Rightarrow \int_0^t f(s)ds=0\quad \forall t\in[0,1]$, pero por el teorema fundamental del cálculo $f(t)=0\quad \forall t\in [0,1]$, lo que implica que $f$ es la función identicamente nula. Lo que nos dice que $Ker(L)=\{0\}$ y por tanto al tener un solo elemento la aplicación es inyectiva.

Para ver que no es sobreyectiva basta ver que no toda función de $C([0,1],\mathbb{R})$ es derivable, por ejemplo el valor absoluto. Y por tanto no pueden ser sobreyectiva.

\item Conectamos con la observación 7. Se puede demostrar, con la ayuda del Lema de Zorn, que todo espacio vectorial admite alguna base. Evidentemente, estamos hablando de base algebraica o base de Hamel. Veamos algunos ejemplos.

	\begin{enumerate}
	\item \textbf{Ejemplo:} $V=\mathcal{P}(\mathbb{R})$, conunto de los polinomios reales. Es claro que $V$ es un e.v real con las operaciones de suma de polinomios y producto por escalares y que una base de $V$ es $B=\{1,x,x^2,\ldots,x^n,\ldots\}$.
	
	Así pues, $V$ es un e.v real de dimensión infinita con base numerable.
	
	\item \textbf{Ejemplo:} $V=C([0,1],\mathbb{R})$, conjunto de funciones continua, definidas en $[0,1]$ y con valores reales. Es claro que $V$ es un e.v real con las operaciones con las operaciones habituales, pero demostraremos con ayuda del Teorema de Categoría de Baire que cualquier base de $V$ es no numerable (al final no se demostró). Así si $B=\{V_\lambda, \lambda\in \Lambda\}$ es cualquier base de $V$, $\Lambda$ es infinito no numerable, o sea que no se puede escribir $\Lambda=\{v_1,v_2,v_3,\ldots\}$, ya que si se puediera sería numerable.
	
	\item \textbf{Ejemplo:} $V=\mathbb{R}^\infty$, conjunto de las sucesiones de números reales, $\mathbb{R}^\infty =\{(\alpha_1,\alpha_2,\alpha_3,\ldots):\alpha_i\in \mathbb{R},\:\forall i\in \mathbb{N}\}$. Es claro que $V$ es un e.v real con las operaciones usuales. Podemos probar que su dimensión es infinita como sigue:
	
	El conjunto
	\begin{gather*}
	e_1=(1,0,0,\ldots)\\
	e_2=(0,1,0,0,\ldots)\\
	e_3=(0,0,1,0,0,\ldots)\\
	\vdots\\
	e_n=(0,\ldots,0^{(n},1,0,0,\ldots),\quad n\in\mathbb{N}
	\end{gather*}
	
	es un conjunto linealmente independiente. Esto prueba que $dim V$ es infinita, pues la base de un espacio es el conjunto con el mayor número de elementos linealmente independientes. Cualquier conjunto de elementos linealmente indpenedientes tiene menos elementos que la base. Tenemos por otro lado que este conjunto no es una base, pues si lo fuera todo elemento del espacio sería combinación lineal (y por tanto finita) de los elementos de la base, pero claramente la sucesión $(1,1,1,\ldots)$ es la suma de todas las sucesiones de la base pero no es un suma finita, luego esta no sería combinación lineal de elementos de la base pero si está en el espacio.
	
	\item \textbf{Ejemplo 4:} $V=\mathbb{R}(\mathbb{Q})$ el conjunto de los números reales, como e.v sobre el cuerpo $\mathbb{Q}$, con las operaciones usuales.
	
	Si $H=\{\alpha_n,\:n\in \mathbb{N}\}$ es cualquier subconjunto de $V$, linealmente independiente, el conjunto de las combinaciones lineales de $H$ es numerable (ya que $\mathbb{Q}$ es numerable). 
	
	Como $\mathbb{R}$ no es numerable (hecho probado por Cantor alrededor de 1871), obtenemos una conclusión: cualquier base de $\mathbb{R}(\mathbb{Q})$ es no numerable.
	\end{enumerate}
	
\item En un e.v, con un producto escalar y dimensión infinita, introduciremos un concepto de base, diferente del de base algebraica o base de Hamel: el concepto de base   Hilbertiana (las bases de Fourier son un buen ejemplo de ellas).

\item Espacios de sucesiones y espacio de funciones, todos ellos e.v de dimensión infinita, que aprecerán a menudo a lo largo de estos apuntes.
	\begin{itemize}
	\item $\forall p\in [1,+\infty)$ definimos
	\begin{equation*}
	l_p=\{(a_n)_{n\in\mathbb{N}}:\:\sum_{n=1}^{+\infty}|a_n|^p<+\infty\}
	\end{equation*}
	\end{itemize}
	
Tomamos sucesiones de elementos de $K$, $K=\mathbb{R}$ ó $K=\mathbb{C}$. $l_p$ es un e.v y quizás la única propiedad no evidente sea la que nos dice que la suma de dos elementos de $l_p$ también esta en $l_p$. Esto no es difícil de probar: 
\begin{gather*}
\forall(a_n)_{n\in\mathbb{N}},\:(b_n)_{n\in\mathbb{N}}
\end{gather*}
, tenemos que
\begin{gather*}
|a_n+b_n|^p\leq (|a_n|+|b_n|)^p\leq (\max\{|a_n|,|b_n|\}+\max\{|a_n|,|b_n|\})^p)=\\
=2^p(\max\{|a_n|,|b_n|\})^p\leq 2^p(|a_n|^p+|b_n|^p)
\end{gather*}

Por tanto, si $\sum_{n=1}^{+\infty} |a_n|^p<+\infty$ y  $\sum_{n=1}^{+\infty}|b_n|^p<+\infty$, tenemos que
\begin{equation*}
\sum_{n=1}^{+\infty} |a_n+b_n|^p<+\infty
\end{equation*}

\textbf{NOTA:} Si $1<p<q<+\infty$, entonces $l_1\subset l_p\subset l_q$. Demuéstrese como ejercicio

\textbf{Demostración:} Supongamos $1\leq p<q$ y sea $x\in l_p$. Pongamos $x=\|x\|_pu$ con $\|u\|_p=1$. Tenemos que
\begin{equation*}
\forall k\in \mathbb{N}\quad |u(k)|\leq \|u\|_p=1\Rightarrow |u(k)|^q\leq |u(k)|^p\Rightarrow \|u\|_q\leq 1
\end{equation*}

Deducimos que $x=\|x\|_pu\in l_q$ y $\|x\|_q=\|x\|_p\|u\|_q\leq \|x\|_p$, esto es, $\|x\|_q\leq\|x\|_p$. Por tanto $l_p\subset \bigcap_{q>p}l_q$, inclusión que es estricta, pues la sucesión $\{x_n\}$ dada por $x_n=\frac{1}{n^{1/p}}$ no está en $l_p$, pero sí está en $l_q$ para todo $q>p$. También tenemos que $\bigcup_{1\leq q<p} l_q\subset l_p$ y la inclusión es estricta, pues la sucesión $\{x_n\}$ dada por $x_n=\frac{1}{n^{1/p}log(n+1)}$ no está en $l_q$ para $1\leq q<p$ y sí está en $l_p$.

Esta segunda inclusión estricta nos dá entre otras cosas que $l_1\subset l_p\subset l_q$. Obteniendo lo pedido.

	\begin{itemize}
	\item Se define $l_\infty=\{(a_n)_{n\in\mathbb{N}}:\:(a_n)_{n\in\mathbb{N}}\:es\:acotada\}$ \\
	
	(Recordemos que $(a_n)_{n\in\mathbb{N}}$ es acotada $\Leftrightarrow \sup_{n\in\mathbb{N}}|a_n|\in \mathbb{R}\Leftrightarrow \sup_{n\in\mathbb{N}}|a_n|<+\infty$)
	
	Aquí $|a_n|$ significa módulo del número complejo $a_n$ (valor absoluto si $a_n$ es real). Trivialmente $l_\infty$ es un e.v sobre $K$ y $l_p\subset l_\infty$, $\forall p\geq 1$
	
	\item Otro ejemplos de e.v son 
	\begin{itemize}
	\item $c_{00}$: sucesiones "casi nulas" ( $(a_n)_{n\in\mathbb{N}}\in c_{00}\Leftrightarrow $ todos sus términos, salvo un número no finito, son nulos)
	\item $c_0$: sucesiones con límite cero.
	\item $c$: sucesiones convergentes
	\end{itemize}
	
	Claramente tenemos $c_{00}\subset c_0\subset c\subset l_{\infty}$, $c_{00}\subset l_1$. Se deja como ejercicio comprobar $c_0\subset l_1$
	
	\item Si $1\leq p\leq \infty $ y $a,b\in \mathbb{R}$, con $a<b$ definimos
	\begin{equation*}
	L^p(a,b)=\{x:[a,b]\rightarrow \mathbb{R}:x\:es\:medible,\quad \int_a^b|x(t)|^p dt<+\infty \}
	\end{equation*}
	donde, desde ahora, todas las integrales que aparecezcan se entenderán en el sentido de Lebesgue.
	
	Recordemos que si, $x,y\in L^p(a,b)$, entonces $x=y\Leftrightarrow x(t)=y(t)$, casi por doquier en $[a,b]$ (en inglés $x(t)=y(t)$, \textit{almost everywhere} en $[a,b]$)$\Leftrightarrow \cancel{\exists} A\subset [a,b]:\mu(A)\neq0,\:x(t)\neq y(t)$ (donde $\mu(A)$ es la medida de Lebesgue de $A$).
	
	\item $L^\infty(a,b)$ se define como
	\begin{equation*}
	L^\infty(a,b)=\{x:[a,b]\rightarrow \mathbb{R},medible,\exists M>0:|x(t)|\leq M,\:c.p.d\:en\:[a,b]\}
	\end{equation*}	
	
	Recordemos que $l_p\subset l_\infty$, $\forall p\in [1,+\infty)$.\\
	
	\textbf{Ejercicio:} Consulta la bibliografía para establecer la relación entre $L^p(a,b),\:L^q(a,b),\:L^\infty (a,b),\quad 1\leq p\leq q$.
	
	\textbf{Solución:} Para $1\leq p < q \leq \infty$ se verifica que $L_q[0,1] \subset L_p[0,1]$ , y $\|f\|_p\leq \|f\|_q$ para toda $f\in L_q[0,1]$. Además, si $1\leq p<q<\infty$, $L_q[0,1]$ es denso en $L_p[0,1]$. \\
	
	\textbf{Demostración:} El caso $q=\infty$ es claro, pues si $f\in L_\infty[0,1]$, entonces $|f(t)|^p\leq \|[f]\|_\infty^p$ casi para todo $x\in [0,1]$, y por tanto $f\in L_p[0,1]$ y $\|f\|_p\leq \|[f]\|_\infty$. Además, la inclusión $L_\infty
	[0,1]\subset L_p[0,1]$ es estricta, pues la función dada por $f(x)=log(\frac{1}{x})$ para $0<x\leq 1$, está en $L_p[0,1]$ cualquiera sea $p\geq 1$, pues la integral

	\begin{equation*}
	\int_0^1\left[log(\frac{1}{x})\right]^p dt=\left[ x=\frac{1}{t}\right] =\int_1^{+\infty} \frac{(log\:u)^p}{u^2} du
	\end{equation*}	
	
	es finita. Pero no está en $L_\infty[0,1]$, pues para $M>0$ se tiene que $\{t\in ]0,1]:|f(t)|>M\}=]0,e^{-M}[$ no es de medida nula.
	
	Sea $1\leq p<q<\infty$ y sea $f\in L_q[0,1]$. Recordemos la desigualdad de Hölder. Si $r>1$ y $s>1$ son tales que $\frac{1}{r}+\frac{1}{s}=1$, y $u\in L_r[0,1]$, $v\in L_s[0,1]$, entonces $uv\in L_1[0,1]$ y 
	\begin{equation*}
	\int_0^1|u(t)v(t)|dt\leq \left(\int_0^1|u(t)|^r\right)^\frac{1}{r}\left(\int_0^1 |f(t)|^q dt\right)^\frac{1}{s}
	\end{equation*}	 
	
	Aplicamos esta desigualdad tomando $u=|f|^p$, $v=\chi_{[0,1]}$, y $r=\frac{q}{p}>1$. Con lo cual $u^r=|f|^q\in L_1[0,1]$, es decir, $u\in L_r[0,1]$, y obtenemos
	\begin{equation*}
	\int_0^1 |f(x)|^p dx\leq \left(\int_0^1 |f(t)|^q dt \right)^\frac{p}{q}
	\end{equation*}
	
	Por tanto, $f\in L_p[0,1]$, y $\|f\|_p\leq \|f\|_q$.\\
	
	La densidad de $L_q[0,1]$ en $L_p[0,1]$ para $1\leq p< q<\infty$, es consecuencia de las funciones continuas $C[0,1]$ son densas en todos los espacios $L_p[0,1]$ con $p\neq \infty$.\\
		
	\item Si $m\in\mathbb{N}$, definimos
	\begin{equation*}
	C^m[a,b]=\{x:[a,b]\rightarrow \mathbb{R}:x,x',\ldots,x^{(m)}\in C[a,b]\}
	\end{equation*}
	
	donde $x^{(m)}$ indica la derivdad de orden $m$ de $x$.
	\end{itemize}
\end{enumerate}

\subsection{COMPLEMENTOS}
\begin{enumerate}
\item Definición de espacio vectorial $V$ sobre un cuerpo $K$. Dos operaciones
\begin{gather*}
+:V\times V\rightarrow V,\quad (x,y)\mapsto x+y\\
\cdot:K\times V\rightarrow V,\quad (\lambda,x)\mapsto \lambda\cdot x
\end{gather*}
$\forall x,y,z\in V$, $\forall\alpha,\beta\in K$, se tiene:
	\begin{enumerate}
	\item Asociativa para $+$
	\begin{equation*}
	x+(y+z)=(x+y)+z
	\end{equation*}
	
	\item $\exists$ elemento neutro para $+$
	\begin{equation*}
	\exists 0\in V:\:x+0=x
	\end{equation*}
	
	\item Conmutativa para $+$
	\begin{equation*}
	x+y=y+x
	\end{equation*}
	
	\item $1\cdot x=x\quad (1\:elemento\:unidad\:de\:K)$.
	
	\item Pseudoasociativa
	\begin{equation*}
	\alpha\cdot(\beta\cdot x)=(\alpha\beta)x
	\end{equation*}
	
	\item Distributiva de los escalares
	\begin{equation*}
	(\alpha+\beta)\cdot x=\alpha\cdot x+\beta \cdot x
	\end{equation*}
	
	\item Distributiva de los vectores
	\begin{equation*}
	\alpha\cdot (x+y)=\alpha\cdot x + \alpha \cdot y
	\end{equation*}
	\end{enumerate}

\item CADINALIDAD DE UN CONJUNTO\\

Dados dos conjuntos $A,B$, se dice que $A$ y $B$ poseen el mismo cardinal (o que son equivalentes en este ambiente) si y sólo si existe alguna aplicación biyectiva $f:A\rightarrow B$. 

Si $A$ y $B$ son conjuntos finitos, entonces $A$ y $B$ son equivalentes si y solo si tienen el "mismo número de elementos". Así, "por convenio" si $A$ y $B$ son conjuntos infinitos equivalentes, decimos que $A$ y $B$ tienen el "mismo número de elementos".

Se denota $\chi_0=Card(\mathbb{N}),\:\chi_1=Car(\mathbb{R}),\:etc.$. Decimos que $Car(A)<Car(B)$ si y sólo si existe alguna aplicación inyectiva de $A$ en $B$, pero ninguna biyectiva. 

Para cualquier conjunto $A$, se tiene
\begin{equation*}
Car(A)<Car(\mathcal{P}(A))
\end{equation*}

donde $\mathcal{P}(A)$ es el conjunto de "todos los subconjuntos de A" (conjunto de las partes de A).\\

\textbf{LEMA DE ZORN:} Sea $A$ un conjunto "parcialmente ordenado" (existe alguna relación de orden en $A$, $\leq$: reflexiva, antisimétrica, transitiva) tal que cualquier $B\subset A$, $B$ "totalmente ordenado" ($\forall x,y\in B$, ó $x\leq y$ ó $y\leq x$), admite alguna cota superior ($\exists x_0\in A/x\leq x_0,\:\forall x\in A$). 

Entonces $A$ tiene algún elemento maximal: $\exists y\in A$ tal que $\cancel{\exists} x\in A$ cumpliento $y<x$ ($\Leftrightarrow y\leq x$, $y\neq x$).\\

Lema de Zorn $\Leftrightarrow$ Axioma de elección
\end{enumerate}

\section{Espacios Normados. Ejemplos}
Sea $X$ un e.v sobre $K(\mathbb{R}$ ó $\mathbb{C})$. Una norma en $X$ es una aplicación $\|\cdot\|:X\rightarrow [0,+\infty)$, $x\mapsto \|x\|$, que satisface:
\begin{enumerate}
\item $\|x\|\geq 0,\:\forall x\in X$, y $\|x\|=0\Leftrightarrow x=0$

\item $\|\lambda x\|=|\lambda|\|x\|$, $\forall \lambda\in K$ y $\forall x\in X$

\item $\|x+y\|\leq \|x\|+\|y\|$, $\forall x,y\in X$ (\textbf{Desigualdad Triangular})
\end{enumerate}

Un espacio vectorial se dice normado si está dotado de una norma $\|\cdot\|$, y se representa como $(X,\|\cdot\|)$.\\

En cualquier espacio normado se puede definir una distancia (o métrica), inducida por la norma:
\begin{equation*}
d:X\times X\rightarrow [0,+\infty),\quad (x,y)\mapsto d(x,y)=\|x-y\|
\end{equation*}

Una vez definido el concepto de norma, espacio normado y distancia veamos algunos ejemplos significativos.

\begin{itemize}
\item \textbf{Ejemplo 1:} $\mathbb{R}^n(\mathbb{R})$, $n\in \mathbb{N}$.

Para todo $p\in [1,+\infty)$, podemos definir
\begin{equation*}
\|x\|_p=(|x_1|+\ldots+|x_n|^p)^\frac{1}{p},\quad \forall (x_1,\ldots,x_n)\in \mathbb{R}^n
\end{equation*}

Es claro que la única propiedad de la norma que no es evidentes es la desigualdad triangular, que en este caso particular es llamada desigualdad de Minkowski. Probaremos esta escalonadamente:\\

\textbf{Desigualdad de Young:} $\forall a,b\in [0,+\infty$, $\forall p>1,\:\forall q>1$ tal que $\frac{1}{p}+\frac{1}{q}=1$, se tiene que
\begin{equation*}
a^{1/p}b^{1/q}\leq \frac{a}{p}+\frac{b}{q}
\end{equation*}

Para demostrar la desigualdad de Young nos basaremos en el siguiente lema.

\textbf{Lema:} $\forall \alpha\in (0,1)$, $\forall x\in [1,+\infty)$, se tiene
\begin{equation*}
\alpha(x-1)+1\geq x^\alpha
\end{equation*}

\textbf{Demostración:} Si $f:[1,+\infty)\rightarrow \mathbb{R}$, $x\mapsto \alpha(x-1)+1-x^\alpha$, entonces
\begin{equation*}
f'(x)=\alpha-\alpha x^{\alpha-1}=\alpha(1-x^{\alpha-1})=\alpha\left(1-\frac{1}{x^{1-\alpha}}\right)\geq 0
\end{equation*}

ya que $x^{1-\alpha}\geq 1$, $\forall x\geq 1$, $\forall \alpha\in (0,1)$. Así $f$ es una función creciente en $[1,+\infty)$. Por tanto $f(x)\geq f(1)=0,\:\forall x\in [1,+\infty)$ y queda por tanto probado el lema. Para probar la \textbf{desigualdad de Young} basta tomar $\alpha=\frac{1}{p}$, $x=\frac{a}{b}$ si $a\geq b$ (si $a\leq b$ basta cambiar a por b) y entonces tenemos

\begin{gather*}
\frac{1}{p}\left(\frac{a}{b}-1\right)+1\geq \left(\frac{a}{b}\right)^\frac{1}{p}\Rightarrow \frac{a-b}{pb}+1 \geq a^\frac{1}{p}b^\frac{-1}{p}\Rightarrow\\
\Rightarrow \frac{a-b}{p}+b\geq a^\frac{1}{p}b^{1-\frac{1}{p}}\Rightarrow \frac{a}{p}+b(1-\frac{1}{p})\geq a^\frac{1}{p}b^{1-\frac{1}{p}}
\end{gather*}

Como $\frac{1}{p}+\frac{1}{q}=1\Rightarrow \frac{1}{q}=1-\frac{1}{p}$, lo que prueba la desigualdad de Young.\\

\textbf{Desigualdad de Hölder:} $\forall x,y\in \mathbb{R}^n$ (ó $\mathbb{C}^n$), tenemos 
\begin{equation*}
\sum_{k=1}^n |x_k\:y_k|\leq \left(\sum_{k=1}^n |x_k|^p\right)^\frac{1}{p}\left(\sum_{k=1}^n |y_k|^q\right)^\frac{1}{q}
\end{equation*}

donde una vez más $\frac{1}{p}+\frac{1}{q}=1$ (Para $p=2$ y $q=2$ obtenemos la desigualdad de Cauchy-Schwarz).\\

\textbf{Demostración:} Si $x=0$ ó $y=0$, la desigualdad es evidente. Así pues, suponemos $x\neq 0$ e $y\neq 0$ ($0\in\mathbb{R}^n$).

Definamos $a_k=\frac{|x_k|^p}{\|x\|_p^p}$, $b_k=\frac{|y_k|^q}{\|y\|_q^q}$, $1\leq k\leq n$. Entonces por la desigualdad de Young tenemos
\begin{gather*}
a_k^{\frac{1}{p}}b_k^\frac{1}{q}\leq \frac{a_k}{p}+\frac{b_k}{q}\Rightarrow \frac{|x_k|}{\|x\|_p}\frac{|y_k|}{\|y_k\|_q}\leq \frac{|x_k|^p}{p\|x\|_p^p}+\frac{|y_k|^q}{q\|y\|_q^q}
\end{gather*}

Así, $|x_k||y_k|\leq \frac{|x_k|^p}{p}\|x\|_p^{1-p}\|y\|_q+\frac{|y_k|^q}{q}\|y\|_q^{1-q}\|x\|_p$. Sumando ahora desde 1 hasta n, tenemos:
\begin{gather*}
\sum_{k=1}^n|x_k||y_k|\leq \frac{1}{p}\|x\|_p^p\|x\|_p^{1-p}\|y\|_q+\frac{1}{q}\|y\|_q^q\|y\|_q^{1-q}\|x\|_p=\\
=\frac{1}{p}\|x\|_p\|y\|_q+\frac{1}{q}\|y\|_q\|x\|_p=\|x\|_p\|y\|_q
\end{gather*}

Con la ayuda de la desigualdad de Hölder, deducimos la desigualdad triangular o la desigualdad de Minkowski para $\|\cdot\|_p$. En efecto, sean $x\in \mathbb{R}^n\backslash \{0\}$ y $y\in \mathbb{R}^n\backslash \{0\}$. Entonces:
\begin{gather*}
\|x+y\|_p^p=\sum_{k=1}^n|x_k+y_k|^p\leq \sum_{k=1}^n(|x_k+y_k|^{p-1})(|x_k|+|y_k|)=\\
=\sum_{k=1}^n|x_k||x_k+y_k|^{p-1}+\sum_{k=1}^n|y_k||x_k+y_k|^{p-1}\leq^{Holder} \\
\leq^{Holder} \left(\sum_{k=1}^n|x_k|^p\right)^\frac{1}{p}\left(\sum_{k=1}^n|x_k+y_k|^{q(p-1)}\right)^\frac{1}{q}+\left(\sum_{k=1}^n|y_k|^p\right)^\frac{1}{p}\left(\sum_{k=1}^n|x_k+y_k|^{q(p-1)}\right)^\frac{1}{q}
\end{gather*}

Como $\frac{1}{p}+\frac{1}{q}=1\Leftrightarrow \frac{1}{q}=1-\frac{1}{p}=\frac{p-1}{p}\Leftrightarrow p=q(p-1)$, tenemos

\begin{gather*}
\|x+y\|^p_p\leq \|x\|_p(\|x+y\|^p_p)^\frac{1}{q}+\|y\|_p(\|x+y\|_p^p)^\frac{1}{q}=(\|x\|_p+\|y\|_p)\|x+y\|_p^\frac{p}{q}
\end{gather*} 

Si $\|x+y\|=0$, la desigualdad triangular es trivial. Finalmente, si $\|x+y\|_p\neq 0$ (o lo que es lo mismo, $x\neq-y$), tenemos:
\begin{equation*}
\|x+y\|_p^{p-\frac{p}{q}}\leq \|x\|_p+\|y\|_p
\end{equation*}

Pero $\frac{1}{p}+\frac{1}{q}=1\Leftrightarrow 1+\frac{p}{q}=p\Leftrightarrow 1=p-\frac{p}{q}$, lo que prueba la desigualdad triangular.

\item \textbf{Ejemplo 2:} $(\mathbb{R}^n,\|\cdot\|_\infty)$, donde $\forall x\in \mathbb{R}^n$, $x=(x_1,\ldots,x_n)$,
\begin{equation*}
\|x\|_\infty=\max\{|x_1|,\ldots,|x_n|\}
\end{equation*}

Es trivial probar las propiedades de la norma.

\item \textbf{Ejemplo 3:} $X=C([a,b],\mathbb{R})$ con $\|\cdot\|_\infty$, donde 
\begin{equation*}
\forall x\in C([a,b],\mathbb{R}),\quad \|x\|_\infty =\max_{t\in[a,b]}|x(t)|
\end{equation*}

El máximo anterior existe, pues toda función real continua definida en un compacto alcanza su máximo y su mínimo.

En $X$ podemos definir otras normas. De hecho, para cada $p,\:1\leq p<+\infty$
\begin{equation*}
\|x\|_p=\left(\int_a^b|x(t)|^p dt\right)^\frac{1}{p}
\end{equation*}

es una norma en $X$.

\item \textbf{Ejemplo 5: "Espacios de Sucesiones"}

Recordemos que si $1\leq p <+\infty$
\begin{equation*}
l_p=\{(a_n)_{n\in\mathbb{N}}:\sum_{n=1}^{+\infty}|a_n|^p<+\infty\}
\end{equation*}

, una norma "natural" para $l_p$ es
\begin{equation*}
\|(a_n)_{n\in\mathbb{N}}\|_p=\left(\sum_{k=1}^{+\infty}|x_k|^p\right)^\frac{1}{p}
\end{equation*}

También 
\begin{equation*}
l_\infty=\{(a_n)_{n\in \mathbb{N}}:(a_n)_{n\in\mathbb{N}}\quad es\:\:acotada\}
\end{equation*}

y una norma "natural" es
\begin{equation*}
\|(a_n)_{n\in \mathbb{N}}\|_\infty=\sup_{n\in\mathbb{N}}|a_n|
\end{equation*}

Las demostraciones para ver que tanto $\|\cdot\|_p$, $1\leq p<+\infty$, como $\|\cdot\|_\infty$, son una norma en los respectivos espacios de sucesiones, son similares al caso $\mathbb{R}^n$ (ejemplos 1 y 2), teniendo la precaución de demostrar primero la "versión de sumas finitas" y luego hacer un paso al límite (no hay problema, pues las series que se consideran son corvergentes).

\item \textbf{Ejemplo 6: "Espacio de Lebesgue de funciones integrables.}

Si $1\leq p<+\infty$
\begin{gather*}
L^p(a,b)=\{x:[a,b]\rightarrow \mathbb{R}:x\quad es\:medible,\int_a^b|x(t)|^p dt<+\infty\}\\
L^\infty(a,b)=\{x:[a,b]\rightarrow \mathbb{R}:x\quad es\:medible,\exists M>0:|x(t)|\leq M\:c.p.d\:en\:[a,b]\}
\end{gather*}
con $\|x\|_p=\left(\int_a^b|x(t)|^p dt\right)^\frac{1}{p},\quad 1\leq p<+\infty,\:\forall x\in L^p(a,b)$, y 
\begin{equation*}
\|x\|_\infty =\inf\{M:|x(t)|\leq M,c.p.d\:en\:[a,b]\},\quad \forall x\in L^\infty(a,b)
\end{equation*}

(este último también llamado supremo esencial de $x$ y notado como $supess|x(t)|$ ó $ess-sup|x(t)|$).\\

La demostración de que las definiciones anteriores constituyen auténticas normas, no es trivial, si $1\leq p<+\infty$ y se puede consultar para ello en la bibliografía recomendada (se recomienda los apuntes de JPerez o de RPaya, en ambos viene demostrado).
\end{itemize}

\subsection{COMPLEMENTOS}
\textbf{1. Espacios métricos.} Sea $X$ un conjunto. Una métrica de $X$ es una aplicación $d:X\times X\rightarrow [0,+\infty)$ que satisface:
\begin{enumerate}
\item $d(x,y)\geq 0$, y $d(x,y)=0\Leftrightarrow x=y$, $\forall x,y\in X$

\item $d(x,y)=d(y,x)$, $\forall x,y\in X$

\item $d(x,z)\leq d(x,y)+d(y,z)$, $\forall x,y,z\in X$ (Desigualdad Triangular).
\end{enumerate}

\textbf{2.} Sea $X$ un espacio vectorial sobre $K$ ($K=\mathbb{R}$ ó $K=\mathbb{C}$). Demuestra que en $X$ se puede definir una norma.\\

\textbf{3.} Sea $X$ un espacio normado y $(x_n)$ una sucesión ocnvergente. Prueba que las sucesiones
\begin{gather*}
\frac{x_1+x_2+\ldots+x_n}{n}\\
\vdots \\
\frac{x_1+2x_2+\ldots+nx_n}{n^2}
\end{gather*}
son convergentes y encuentra su límite.\\

\textbf{Solución:} Sean $(a_n)_{n\in \mathbb{N}}=\sum_{i=1}^n\|x_i-x\|$ y $(b_n)_{n\in\mathbb{N}}=n$
\begin{gather*}
\lim_{n\to \infty}\frac{a_n-a_{n-1}}{b_n-b_{n-1}}=\lim_{n\to\infty}\|x_n-x\|=0\Rightarrow^{Criterio\:Stolz}\lim_{n\to \infty}\frac{\sum_{i=1}^n\|x_i-x\|}{n}=0\Rightarrow\\
\Rightarrow \lim_{n\to\infty}v_n=\lim_{n\to\infty}\frac{x_1+\ldots+x_n}{n}=x
\end{gather*}

\textbf{4.} Desigualdad de Young, Hölder y Minkowski para funciones. (Vienen hechas en los apuntes de Jperez).

\section{Topología de Espacios Normados}
Si $(X,\|\cdot\|)$ es un espacio normado, entonces en $X$ se puede definir una métrica $d:X\times X\rightarrow \mathbb{R}$, $(x,y)\mapsto d(x,y)=\|x-y\|$. Por tanto cualquier espacio normado es espacio métrico y consecuentemente, espacio topológico.

Eso nos permite hablar de bolas abiertas, bolas cerradas, subconjuntos abiertos, cerrados, interior, clausura, densidad, sucesiones y series convergentes, series de Cauchy, espacios completos, etc.\\

En este tema nos vamos a concentrar en los conceptos y propiedades "propios de la dimensión infinita" y las diferencias fundamentales con los espacios de dimensión finita ($\mathbb{R}^n$ ó $\mathbb{C}^n$), con los que ya estamos familiarizados.

\begin{enumerate}
\item $(X,\|\cdot\|)\Rightarrow (X,d)$ métrico ($d(x,y)=\|x-y\|)$. Sin embargo, $(X,d)\cancel{\Rightarrow}$ la métrica derive de una norma.

\textbf{Ejemplo:} $d:X\times X\rightarrow \mathbb{R}$ dada por
\begin{gather*}
d(x,y)=\left\lbrace \begin{array}{c}
1\quad si\:x\neq y\\
0\quad si\:x=y
\end{array}\right.
\end{gather*}

\item Si $(X,\|\cdot\|)$ es normado, las aplicaciones:
\begin{enumerate}
\item $X\times X\rightarrow X$, $(x,y)\mapsto x+y$

\item $K\times X\rightarrow X$, $(\lambda,x)\mapsto \lambda x$

\item $X\rightarrow \mathbb{R}$, $x\mapsto \|x\|$
\end{enumerate}

son continuas, cuando en $X\times X$  en $K\times X$ se considera la topología producto.

\textbf{NOTA:} En $X\times X$ podemos definir, por ejemplo, la norma: $\|(x,y)\|=\max\{\|x\|,\|y\|\}$. Entonces la bola abierta de centor $(a,b)\in X\times X$ y radio $r>0$, en $X\times X$, satisface:
\begin{equation*}
B_{X\times X}((a,b),r)=B_X(a,r)\times B_X(b,r)
\end{equation*}

luego la topología de la norma $\|(\cdot,\cdot)\|$ en $X\times X$ es la topología producto en $X\times X$. Análogamente $K\times X$.\\

Probamos ya la continuidad de las aplicaciones anteriores:

\textbf{(a)} Si $(x_n,y_n)\rightarrow (x,y)$, entonces $x_n\rightarrow x$ y $y_n\rightarrow y$. Como $\|x_n+y_n-(x+y)\|\leq \|x_n-x\|+\|y_n-y\|$, tenemos que $(x_n+y_n)\rightarrow x+y$.\\

\textbf{(b)} Si $(\lambda_n,x_n)\rightarrow (\lambda,x)\in K\times X$, entonces $(\lambda_n)\rightarrow \lambda$ y $(x_n)\rightarrow x$. Como $(x_n)\rightarrow x$, la sucesión $(x_n)$ está acotada: $\exists M>0/\|x_n\|\leq M,\:\forall n\in \mathbb{N}$. Luego
\begin{gather*}
|\lambda_n x_n-\lambda x|=|\lambda_n x_n-\lambda x_n+\lambda x_n-\lambda x| \leq |\lambda_n-\lambda|\|x_n\|+|\lambda|\|x_n-x\|\leq \\
\leq |\lambda_n-\lambda|M+|\lambda|\|x_n-x\|\rightarrow 0
\end{gather*}

\textbf{(c)} La demostración es trivial si usamos la desigualdad
\begin{equation*}
|\|x\|-\|y\||\leq \|x-y\|,\quad \forall x,y\in X
\end{equation*}

ya que si $x_n\rightarrow x$, entonces, $\|x_n\|\rightarrow \|x\|$, basta con llamar a $x=x_n$ y a $y=x$ y tender la n a infinito, tendríamos que la desigualdad está acotada superiormente por 0, y por tanto cuando n tiende a infinito $\|x_n\|\rightarrow \|x\|$.


\item "Diferencia importante entre dimensión finita e infinita"

Vamos a comparar la convergnecia de $\mathbb{R}^2$ con la de $l_\infty$ (pero lo mismo ocurriría con $\mathbb{R}^p$ y $l_q$, $1\leq q\leq +\infty$).\\

Sea $(x^n)$ una sucesión de $\mathbb{R}^2$, $x^n=(x^n_1,x^n_2)$, $\forall n\in \mathbb{N}$ y $x=(x_1,x_2)\in \mathbb{R}^2$. Entonces, si en $\mathbb{R}^2$ consideramos la norma $\|y\|_{\mathbb{R}^2}=\|(y_1,y_2)\|_{\mathbb{R}^2}=\max\{|y_1|,|y_2|\}$ se tiene:
\begin{equation*}
x^n\xrightarrow{\mathbb{R}^2}x\Leftrightarrow \left\lbrace \begin{array}{c}
x_1^n\xrightarrow{n\to\infty} x_1\\
x_2^n\xrightarrow{n\to\infty} x_2
\end{array}\right.
\end{equation*}

Esto es fácil de probar, pues:
\begin{gather*}
(x^n)\xrightarrow{\mathbb{R}^2} x\Leftrightarrow \|(x^n-x)\|_{\mathbb{R}^2}\xrightarrow{n\to\infty} 0\Leftrightarrow\\
\Leftrightarrow \max\{|x_1^n-x_1|,|x_2^n-x_2|\}\xrightarrow{n\to\infty}0 \Leftrightarrow\\
\Leftrightarrow x_1^n\xrightarrow{n\to\infty}x_1,\quad x_2^n\xrightarrow{n\to\infty} x_2
\end{gather*}

\textbf{Atención:} En $\mathbb{R}^2$ podemos considerar cualquier otra norma (ya que en $\mathbb{R}^n$ cualesquieran dos normas son equivalentes). Podemos tratar con $\mathbb{R}^p$, con $p\in \mathbb{N}$ y el resultado es el mismo (bastaría de nuevo usar la norma uniforme, sería análogo a lo visto, independientemente de la dimensión y si luego queremos aplicarlo a otra norma basta usar que será equivalente a la norma del máximo).

Si $(f^n)$ es una sucesión en $\mathbb{R}^p$:
\begin{gather*}
f^1=(f_1^1,f_2^1,\ldots,f_p^1)\\
f^2=(f^2_2,f^2_2,\ldots,f_p^2)\\
\cdots\\
f^n=(f_1^n,f_2^n,\ldots,f_p^n)
\end{gather*}

entonces
\begin{gather*}
(f^n)\xrightarrow{\mathbb{R}^p}f=(f_1,\ldots,f_p)\Leftrightarrow (f_i^n)\xrightarrow{n\to\infty}f_i,\quad 1\leq i\leq p
\end{gather*}

Ahora veamos el siguiente ejemplo en $l_\infty$:
\begin{gather*}
f^1=(0,1,1,1,\ldots)\\
f^2=(0,0,1,1,1,\ldots)\\
f^3=(0,0,0,1,1,1,\ldots)\\
\cdots\\
f^n=(0,\ldots^{(n},0,1,1,1,\ldots)\\
\cdots
\end{gather*}

la sucesión $(f^n)$ es una sucesión de elementos de $l_\infty$ que, por columnas converge al elemento
\begin{equation*}
f=(0,0,0,0,\ldots\in l_\infty
\end{equation*}

pero $(f^n)\cancel{\xrightarrow{l_\infty}} f$, pues $\|f^n\|_\infty=1$, $\forall n\in \mathbb{N}$.\\

Sabemos que convergencia en $l_p$, con $1\leq p\leq \infty$ implica convergencia por columnas, pero el recíproco no es cierto.

\item Espacios de Banach. Ejemplos importantes.
\end{enumerate}

Un grupo especial de espacios normados lo constituyen los "espacios normados completos" (aquellos donde cualquier sucesión de Cauchy es convergente a un elemento del espacio normado considerado). El ejemplo más elemental es $\mathbb{R}^n$ (con cualquier norma), $\mathbb{C}^n$, etc. (en general, cualquier espacio normado de dimensión finita). En dimensión infinita es más interesante, pues no encontraremos espacios normados completos (de Banach) y otro que no lo sean como se ve en los siguientes ejemplos:

\begin{enumerate}
\item $l_p$, $1\leq p\leq +\infty$, con la norma habitual, es un espacio de Banach. Veamos la demostración.\\

\textbf{Demostración:} Vamos a demostrar, por simplicidad, que $(l_2,\|\cdot\|_2)$ es completo, y el resto de casos serán análogos a las ideas usadas en este. Recordemos

\begin{gather*}
l_2=\{(x_n)_{n\in\mathbb{N}}:\sum_{n=1}^{+\infty}|x_n|^2<+\infty\}\\
\|(x_n)\|=\left(\sum_{n=1}^{+\infty}|x_n|^2\right)^\frac{1}{2},\quad \forall (x_n)\in l_2
\end{gather*}

Sea $(x^k)_{k\in\mathbb{N}}$ una sucesión de Cauchy en $l_2$, vamos a demostrar que si es de Cauchy es convergente a un elemento del espacio. En efecto, pensemos que $(x^k)_{k\in\mathbb{N}}$ es una "sucesión de sucesiones":
\begin{gather*}
x^1=(x^1_1,x_2^1,\ldots,x_n^1,\ldots)\\
x^2=(x^2_1,x_2^2,\ldots,x_n^2,\ldots)\\
\cdots\\
x^k=(x^k_1,x_2^k,\ldots,x_n^k,\ldots)
\end{gather*}

Como $(x^k)$ es de Cauchy en $l_2$:
\begin{equation*}
\forall \varepsilon >0,\:\exists k_0(\varepsilon)\in\mathbb{N}\::\left.\begin{array}{c}
\forall k\geq k_0(\varepsilon)\\
\forall p\in\mathbb{N}
\end{array}\right.\:\|x^{k+p}-x^k\|_2\leq \varepsilon
\end{equation*}

Como $\|x^{k+p}-x^k\|_2=\left(\sum_{n=1}^{+\infty}|x_k^{k+p}-x_n^k|^2\right)^\frac{1}{2}\leq \varepsilon$, esto implica:
\begin{equation*}
\forall \varepsilon >0, \exists k_0(\varepsilon)\in\mathbb{N}:\left.\begin{array}{c}
\forall k\geq k_0(\varepsilon)\\
\forall p\in\mathbb{N}\\
\forall n\in\mathbb{N}
\end{array}\right\rbrace,\:|x_n^{k+p}-x_n^k|\leq \varepsilon
\end{equation*}
(pensemos que $\forall n\in\mathbb{N}$, fijo, $|x_n^{k+p}-x_n^k|\leq \|x^{k+p}-x^k\|$).\\

\textbf{Importante:} Observemos que la desigualdad anterior es "uniforme" respecto de $p\in \mathbb{N}$, $n\in \mathbb{N}$. 

\begin{itemize}
\item Fijado $n\in\mathbb{N}$, la desigualdad implica que la sucesión $(x_n^k)_{k\in\mathbb{N}}$ es una sucesión de Cauchy de números reales (o complejo). Es decir, cada columna de la expresión es una sucesión de Cauchy (supongamos real, aunque es similar si es compleja). La conclusión es que $\forall n\in\mathbb{N}$ fijo, $x_n^k\xrightarrow{k\to\infty}x_n$.

Sea $x=(x_n)_{n\in\mathbb{N}}$. Buscamos ver que nuestra sucesión de sucesiones converge a $x$. La clave está en lo siguiente:
\begin{equation*}
\forall \varepsilon >0,\:\exists k_0(\varepsilon)\in\mathbb{N}:\:\left.\begin{array}{c}
\forall k\geq k_0(\varepsilon)\\
\forall p\in\mathbb{N}
\end{array}\right. 
\end{equation*}
tenemos,
\begin{gather*}
\sum_{n=1}^N|x_n^{k+p}-x_n^k|^2\leq \varepsilon^2,\quad \forall n\in\mathbb{N}\Rightarrow\\
\Rightarrow \sum_{n=1}^N|x_n-x_n^k|^2\leq \varepsilon^2,\left.\begin{array}{c}
\forall N\in\mathbb{N}\\
\forall k\geq k_0(\varepsilon)
\end{array}\right.\Rightarrow\\
\Rightarrow \sum_{n=1}^{+\infty}|x_n-x_n^k|^2\leq \varepsilon^2 \quad \forall k\geq k_0(\varepsilon)
\end{gather*}

, pero lo anterior nos indica que $x-x^k\in l_2$, $\forall k\geq k_0(\varepsilon)$. Además $x=x^k+(x-x^k)\Rightarrow x\in l_2$ y fijándonos bien tenemos que:
\begin{equation*}
\forall \varepsilon>0,\:\exists k_0(\varepsilon)\in\mathbb{N}:\:\forall k\geq k_0(\varepsilon)\Rightarrow\|x-x^k\|_2\leq \varepsilon
\end{equation*}

\item Sea $(\mathcal{P}_{[0,1]},\|\cdot\|_\infty)$ el espacio normado de funciones polinomiales (con coeficientes reales), restringidos a $[0,1]$, con la norma $\|f\|_\infty=\max_{t\in[0,1]}|f(t)|,\:\forall f\in\mathcal{P}_{[0,1]}$.

Demostremos que no es un espacio de Banach. Para ello recordemoas:
\begin{equation*}
e^x=1+\frac{x}{1!}+\frac{x^2}{2!}+\cdots+\frac{x^n}{n!}+\cdots
\end{equation*}

de manera uniforme en $[0,1]$. Pero de esto anterior se deduce que la sucesión de polinomios $(P_n)_{n\in\mathbb{N}}$,
\begin{equation*}
P_n(x)=1+\frac{x}{1!}+\frac{x^2}{2!}+\cdots+\frac{x^n}{n!}
\end{equation*}

converge en $(\mathcal{P}_{[0,1]},\|\cdot\|_\infty)$ a la función exponencial. Luego $(P_n)_{n\in\mathbb{N}}$ es una sucesión de Cauchy en $(\mathcal{P}_{[0,1]},\|\cdot\|_\infty)$ que no converge en $(\mathcal{P}_{[0,1]},\|\cdot\|_\infty)$.

\item Otra diferencia importante entre los espacios normados de dimensión finite e infinita.\\

Si $(X,\|\cdot\|)$ es un espacio normado, cualquier subespacio vectorial $F$, de $X$ de dimensión finita es cerrado. Esto es "sencillo" y, básicamente es un "juego de palabras" llevado a cabo convenientemente. En efecto, sea $B=\{x^1,x^2,\ldots,x^m\}$ una base de $F$. Entonces $\forall f\in \overline{F}$ (clausura o cierre de $F$), tenemos que $\exists (f^k)_{k\in\mathbb{N}}\subset F/f^k\xrightarrow{k\to\infty}f$.

Si escribimos
\begin{gather*}
f^1=a_1^1x^1+a_2^1x^2+\cdots+a^1_mx^m\\
f^2=a_1^2x^1+a_2^2x^2+\cdots+a^2_mx^m\\
\cdots\\
f^k=a_1^kx^1+a_2^kx^2+\cdots +a_m^kx^m
\end{gather*}

entonces $\exists a\in \mathbb{R}^m$ tal que
\begin{equation*}
\left\lbrace\begin{array}{c}
(a_1^k)\xrightarrow{k\to\infty} a^1\\
(a_2^k)\xrightarrow{k\to\infty} a^2\\
\cdots\\
(a^k_m)\xrightarrow{k\to\infty}a^m
\end{array}\right.
\end{equation*}

En $F$ tenemos definida la norma $\|\cdot\|$, inducida por $X$. Pero como $F$ tiene dimensión finita, todas las normas definidas en $F$ son equivalentes y si usamos por ejemplo, $\|\cdot\|_\infty$ en $F$, es decir
\begin{equation*}
\|g\|_{\infty}=\max_{1\leq i\leq m}|\lambda_i|,\quad \forall g=\lambda_1x^1+\cdots+\lambda_m x^m\in F
\end{equation*}

sabemos que la convergencia de la sucesión implica la convergencia por columnas. Conclusión $f^k\xrightarrow{\|\cdot\|}a^1x^1+\cdots+a^mx^m\in F$. Así $\overline{F}=F$ y $F$ es cerrado.
\end{itemize}
\end{enumerate}


En dimensión infinita, la situación puede cambiar drásticamente. Para ello veamos un ejemplo:\\

$c_{00}$ es denso en $l_2$ (es decir, $\overline{c_{00}}=l_2$, en el espacio $(l_2,\|\cdot\|_2)$.

Observemso que $c_{00}$ (conjunto de las sucesiones "casi nulas") es un subespacio vectorial de $l_2$. Claramente, $\overline{c_{00}}\subset l_2$. Veamos el recíproco: Sea $x\in l_2$, $x=(x_n)_{n\in\mathbb{N}}$. Como la serie $\sum_{n=1}^{+\infty} |x_n|^2$ es convergente, la serie de restos tiende a cero. Es decir la sucesión
\begin{equation*}
\sum_{n=1}^{+\infty}|x_n|^2,\sum_{n=2}^{+\infty}|x_n|^2,\ldots,\sum_{n=k}^{+\infty}|x_n|^2,\ldots
\end{equation*}

tiende a cero cuando $k\to+\infty$. Por tanto:
\begin{equation*}
\forall \varepsilon>0,\:\exists k_0(\varepsilon)\in \mathbb{N}:\:\forall k\geq k_0,\:se\:tiene\:\sum_{n=k}^{+\infty}|x_n|^2\leq \varepsilon
\end{equation*}

Consideremos el elemento de $c_{00}$ dado por
\begin{equation*}
y^{k_0}=(x_1,x_2,\ldots,x_{k-1},0,0,\ldots)
\end{equation*}

Entonces $\|x-y^{k_0}\|_2=\left(\sum_{n=k_0}^{+\infty}|x_n|^2\right)^\frac{1}{2}\leq \varepsilon^\frac{1}{2}$.

El razonamiento anterior prueba que $\overline{c_{00}}=l_2$.\\

\textbf{Noción de separabilidad}\\

Un espacio normado $(X,\|\cdot\|)$ se dice separable si existe algún subconjunto $D\subset X$, $D$ denso y numerable.

Por ejemplo, $\mathbb{R}^n$ es separable, pues podría tomar $D=\mathbb{Q}^n$.

Una idea parecida puede usarse para ver que $l_p$, $1\leq p<+\infty$ es separable, pero $l_\infty$ no lo es. En efecto, demostremos que $l_1$ es seprable (las mismas ideas se usan para $l_p$, $1<p<+\infty$).\\

Sea $D\subset l_1$, $D$ es el conjunto de las sucesiones racionales casi nulas ($D$ es la unión numerable de conjuntos numerables, y por tanto $D$ es numerable). Por otra parte, $\forall x\in l_1$, $\forall \varepsilon >0$, $\exists k_0(\varepsilon)$:$k\geq k_0(\varepsilon)$, se tiene $\sum_{n=k}^{+\infty}|x_n|<\varepsilon$.

como $\mathbb{Q}$ es denso en $\mathbb{R}$, $\exists q_1,\ldots,q_{k_0}\in\mathbb{Q}$ tal que $\sum_{n=1}^{k_0}|x_n-q_n|<\epsilon$. Por tanto si $y=(q_1,\ldots,q_{k_0},0,0,\ldots)$ tenemos que $\|x-y\|_1\leq 2\varepsilon$. Esto prueba que $\overline{D}=l_1$.\\

En cambio, $l_\infty$ no es seprable. Para ver que no es separable, podemos proceder como sigue:

Sea $C=\{x^p:p\in\mathbb{N}\}\subset l_\infty$, denso. Sea $B\subset l\infty$ y 
\begin{equation*}
B=\{(y_n)_{n\in\mathbb{N}}:y_n\in\{0,1\},\:\forall n\in \mathbb{N}\}
\end{equation*}

Notemos que si $y=(y_n)_{n\in \mathbb{N}}$, $z=(z_n)_{n\in\mathbb{N}}$ son elementos distintos de $B$, entonces $\|y-z\|_{\infty}=1$. Además, como $C$ es denso en $l_\infty$, $\forall y\in B$ $\exists x^{p(y)}\in C$:$\|y-x^{p(y)}\|_{\infty}<\frac{1}{4}$. Claramente $B_{l_\infty}(y,\frac{1}{3})\cap B_{l_\infty}(z,\frac{1}{3})=\emptyset$, $\forall x,y\in B$, $y\neq z$ (pues $\|y-z\|_\infty=1$).

Tenemos que la aplicación $\left.\begin{array}{c}
B\rightarrow \mathbb{N}\\
y\mapsto p(y)
\end{array}\right.$ es inyectiva. Luego $B$ es numerable. Pero sabemos que $B$ no es numerable (pensar que cualquier número real se puede representar en binario).(JPerez tiene en sus apuntes otra demostración).\\

Si $(X,\|\cdot\|)$ es un espacio normado tal que $\exists B\subset X$, $B$ no numerable y satisfaciendo la propiedad siguiente
\begin{equation*}
\exists\alpha>0:\:\: \|x-y\|\geq \alpha,\:\:\forall x,y\in B,\:\:x\neq y
\end{equation*}

entonces $(X,\|\cdot\|)$ no es separable.\\

\textbf{Compacidad}\\

Sea $(X,\|\cdot\|)$ un espacio normado y $B\subset X$. Como $X$ es también un espacio métrico (la métrica derivada de la norma) y por tanto, un espacio topológico, tiene sentido hablar de compacidad: $B$ es compacto cuando para cualquier recubrimiento de $K$, por abierto, es posible extraer un subrecubrimiento finito. Esta definición es poco útil en la práctica. Nos interesan más los siguientes:
\begin{itemize}
\item Si $dim(X)$ es finita, $B$ es compacto si y sólo si $B$ es cerrado y acotado.

\item En espacio métricos (y por tanto en espacios normados) la compacidad es esquivalente a la "compacidad secuencial" (o "compacidad por sucesiones"):
\begin{equation*}
B\:\:es\:\:compacto \Leftrightarrow \forall (x_n)_{n\in\mathbb{N}}\subset B,\:\:\exists (x_{n_k})\rightarrow x\in B
\end{equation*}

Por tanto, como normalmente trabajamos con espacios normados, vamos a adoptar la anterior definición como "definición de conjunto compacto".
\end{itemize}

\textbf{Ejercicio.} Si $(X,\|\cdot\|)$ es normado y $B\subset X$ es compacto, demuestra que $B$ es cerrado y acotado.\\

En efecto, $\forall x\in \overline{B}$(clausura o cierre de $B$), $\exists (x_n)\subset B:\:(x_n)\rightarrow x$. Como $B$ es compacto, $\exists (x_{n_k})\rightarrow y\in B$. Ahora bien $(x_{n_k})$ es una subsucesión de $(x_n)$, que es convergente a $x$. Luego $y=x$, lo que implica $x\in B$. Por tanto $\overline{B}=B$, y $B$ es cerrado. Por otra parte, si $B$ no fuese acotado $\exists (x_n)\subset B:\|x_n\|>n,\:\forall n\in \mathbb{N}$. Esto implica que $(x_n)$ no puede tener ninguna sucesión parcial convergente, pues cualquier sucesión convergente es acotada y $(x_{n_k})$ no lo es.\\

\textbf{Ejercicio.} Demuestra que la bola unidad de $l_2$ no es compacta.\\

Sea $(x_n)_{n\in \mathbb{N}}$ la sucesión de "vectores canónicos de $l_2$". Entonces $\|x_n-x_m\|=\sqrt{2},\:\forall n\neq m$. Esto implica que cualquier sucesión parcial de $(x_n)$ no es de Cauchy. Así $(x_n)$ no puede tener ninguna sucesión parcial convergente.\\

\textbf{Ejercicio.} Demuestra que la bola cerrada unidad de $(C[0,1],\|\cdot\|_\infty)$ no es compacta.\\

Sea $(x_n)_{n\in\mathbb{N}}$ dada por $x_n(t)=t^n$, $\forall t\in[0,1]$, $\forall n\in\mathbb{N}$. Si $(x_{n_k})$ es alguna subsucesión convergente a $x\in C[0,1]$, entonces $x_{n_k}(t)=t^{n_k}$. Como $n_k\to +\infty$, $x_{n_k}$ converge puntualmente a la función
\begin{equation*}
y:[0,1]\rightarrow \mathbb{R},\quad t\rightarrow \left\lbrace \begin{array}{c}
0,\quad t\in [0,1]\\
1,\quad t=1
\end{array}\right.
\end{equation*}

Como $y$ debe ser igual a $x$ (por la unicidad del límite), esto no es posible, pues $y\notin C[0,1]$.

\subsection{Complementos}
\textbf{Ejercicio:} $(C[0,1],\mathbb{R}),\|\cdot\|_{\infty})$ es separable.

Este ejercicio se basa en que por el Teorema de Aproximación de Weierstrass toda función real continua definida en un intervalo cerrado y acotado puede ser aproximada tanto como se quiera por un polinomio de coeficientes reales. Pero todo polinomio de coeficientes reales puede ser aproximado tanto como se quiera por un polinomio de coeficientes racionales, luego toda función continua se puede aproximar tanto como se quiera por un polinomio de coeficientes racionales, de lo que se deduce que el subconjunto de los polinomios de coeficientes racionales es un subconjunto denso de $(C[0,1],\mathbb{R}),\|\cdot\|_{\infty})$. Por último, es claro que el subconjunto de los polinomios de coeficientes racionales es numerable, luego se tiene lo pedido.

\section{Operadores Lineales}
En Análisis Funcional, los operadores lineales juegan un papel muy importante. El alumnado está familiarizado con el caso finito dimensional. De hecho, si $L:\mathbb{R}^n\rightarrow \mathbb{R}^m$ es lineal, es decir,
\begin{equation}
L(\alpha x+\beta y)=\alpha L(x)+\beta L(y),\quad \forall \alpha,\beta\in\mathbb{R},\:\forall x,y\in\mathbb{R}^n
\end{equation}
entonces $L$ se representa como 
\begin{equation}
L(x)=Ax,\quad\forall x\in\mathbb{R}^n
\end{equation}

donde $A$ es una matriz real de $n\times m$, y recíprocamente, cualquier operador de la forma (2), es un operador lineal. Tirivalmente, en este caso, cualquier operador lineal es continuo (es más, veremos que con que la dimensión del espacio de partida sea finita, entones todo operador lineal es continuo). 

Si $(E,\|\cdot\|_E)$ y $(F,\|\cdot\|_F)$ son espacios normados, $L:E\rightarrow F$ se dice lineal si satisface 
\begin{equation*}
L(\alpha f+\beta g)=\alpha L(f)+\beta L(g),\quad \forall \alpha,\beta \in\mathbb{R},\:\forall f,g\in E
\end{equation*}

Ejemplos de operadores lineales son:
\begin{enumerate}
\item $L:(C^1[0,1],\|\cdot\|_\infty)\rightarrow (C^0[0,1],\|\cdot \|_\infty),\quad f\mapsto f'$

\item $L:(C^1[0,1],\|\cdot\|_{\infty})\rightarrow \mathbb{R},\quad f\mapsto \int_0^1 f(t)dt$

\item $L:(C^1[0,1],\|\cdot\|_1)\rightarrow (C[0,1],\|\cdot \|_{\infty}),\quad f\mapsto\int_0^1 f(t)dt+f'(t)$
\end{enumerate}

Puede ocurrir que, como en algún ejemplo anterior, $L$ no sea continuo (concretamente el caso 1 no es continuo).

Por tanto, el tema de los operadores lineales, cuando entren en juegos espacios normados de dimensión infinita no es trivial. Comencemos con algunas propiedades de los operadores lineales.\\

Sean $(E,\|\cdot\|_E)$, $(F,\|\cdot\|_F)$ espacios normados y $L:E\rightarrow F$ lineal. Entonces:
\begin{enumerate}
\item $L$ es continuo $\Leftrightarrow$ $L$ es continuo en $e=0$.

\textbf{Demostración}:$\Rightarrow)$ Evidente.

$\Leftarrow)$ Si $e_0\in E$ y $(e_n)\xrightarrow{n\to\infty}e_0$ entonces $(e_n-e_0)\xrightarrow{n\to\infty}0\Rightarrow L(e_n-e_0)\rightarrow L(0)=0\Rightarrow L(e_n)-L(e_0)\rightarrow 0$.\\

\item $L$ es continuo $\Leftrightarrow L$ es continuo en algún $e_0\in E$.

\textbf{Demostración:}$\Rightarrow)$ Evidente.

$\Leftarrow)$ Sea $e\in E$ y $(e_n)\rightarrow e$. Entonces $e_0+(e_n-e)\Rightarrow L(e_0+e_n-e)$ converge a $L(e_0)$. Ahora bien $L(e_0+e_n-e)=L(e_0)+L(e_n)-L(e)\rightarrow L(e_0)\Rightarrow L(e_n)\rightarrow L(e)$.

\item $L$ es continuo $\Leftrightarrow$ $\exists k\geq 0$ tal que $\|L(e)\|_F\leq k\|e\|_E,\:\:\forall e\in E$.\\

\textbf{Demostración:} $\Leftarrow)$ Si $e\in E$ y $(e_n)\rightarrow e$, entonces $\|L(e_n)-L(e)\|=\|L(e_n-e)\|\leq k\|e_n-e\|\rightarrow 0$. Por tanto, $L(e_n)\rightarrow L(e)$.\\

$\Rightarrow)$ Si no fuese cierta la implicación, entonces $\forall n\in \mathbb{N}$, $\exists e_n\in E\backslash\{0\}$ tal que $\|L(e_n)\|>n\|e_n\|\Rightarrow \|L(\frac{e_n}{n\|e_n\|}\|>1$. Pero $(\frac{e_n}{n\|e_n\|})\rightarrow 0$, pues $\|\frac{e_n}{n\|e_n\|}\|=\frac{1}{n}$. Por tanto, por hipótesis, $L(\frac{e_n}{n\|e_n\|})\rightarrow L(0)=0$, luego llegamos a una contradicción.

\item $L$ es continuo $\Leftrightarrow \sup_{e\in E\backslash\{0\}}\frac{\|L(e)\|}{\|e\|}<+\infty\Leftrightarrow \sup_{\|e\|=1}\|L(e)\|<+\infty\Leftrightarrow \forall A\subset E$, $A$ acotado en $E\Rightarrow L(A)$ es acotado en $F$.

La demostración de las equivalentcias anteriores es trivial, teniendo en cuenta lo ya demostrado.
\end{enumerate}

Como se ve, hay muchas caracterizaciones sobre la continuidad de los operadores lineales, pero recordad que es para operadores LINEALES, para operadores no lineales no se tienen porque verificar.\\

\textbf{Ejercicio.} Sea $L:(E,\|\cdot\|_E)\rightarrow (F,\|\cdot\|_F)$ lineal, y $dim(E)$ finita. Entonces $L$ es continua.\\

\textbf{Demostración:} Sea $L_1:E\rightarrow Img(L)$. Como la dimensión de $E$ es finita también lo será la de su imagen por $L$, luego esta función es una función entre espacios de dimensión finita y por tanto continua. Si componemos esta con la inclusión de $Img(L)$ en $F$ tenemos
\begin{equation*}
L=i\circ L_1
\end{equation*}
, luego se tiene que $L$ es continua por ser composición de continuas.\\

\textbf{Ejercicio.} Sea $(E,\|\cdot\|_E)$ tal que $dim(E)$ es infinita. Demuestra que existe $L:E\rightarrow E$ lineal tal que $L$ no es continua, y $M:E\rightarrow \mathbb{R}$ lineal, no continuo.\\

\textbf{Solución:} Si $dim(E)$ es infinita, simepre existe $L:E\rightarrow \mathbb{R}$ lineal no continua.

Sea $B$ una base de $E$, $B=\{V_\lambda:\lambda\in \Lambda\}$, donde $\Lambda$ es conjunto infinito $\exists B'=\{V_1,V_2,\ldots\}\subset B$.

Necesitamos definir $\{L(V_\lambda),\lambda\in\Lambda \}$ y "extender por linealidad". Por último
\begin{equation*}
L(v)=\left\lbrace \begin{array}{c}
n\quad si\:\:v=v_n\in B'\\
0\quad si\:\:v\in B\backslash B'
\end{array}\right. \Leftrightarrow 
\end{equation*}
$L$ no está acotada en $\overline{B}(0,1)$ y queda demostrado que no es continua.\\

La otra demostración es igual solo que para que $L$ vaya de $E$ en $E$ basta poner que $L(v)=ny$, $\forall v\in B'$ y con $y\in E\backslash\{0\}$.\\

Antes de enunciarlo, recordemos que un hiperplano en $\mathbb{R}^n$ viene dado por conjuntos del tipo:
\begin{equation*}
B=\{(x_1,\ldots,x_n)\in \mathbb{R}^n:a_1x_1+\ldots+a_nx_n=c\}
\end{equation*}

donde $(a_1,\ldots,a_n)\in \mathbb{R}^n\backslash \{0\}$, $c\in\mathbb{R}$, son dados.

Por ejemplo, en $\mathbb{R}^2$, los hiperplanos son rectas, en $\mathbb{R}^3$ los hiperplanos son planos, etc.(otra forma de verlo en espacios de dimensión finita, es que los hiperplanos son espacios de dimensión uno menos que la dimensión del espacio original).\\

\textbf{Ejercicio.} Sea $(X,\|\cdot\|)$ un espacio normado de dimensión infinita. Un hiperplano se define como un conjunto de la forma
\begin{equation*}
H=\{x\in X:f(x)=\alpha\}
\end{equation*}

donde $f:X\rightarrow \mathbb{R}$ es lineal, no idénticamente cero y $\alpha\in\mathbb{R}$, dados.

Demuestra que cualquier hiperplano de $H$ es o cerrado, o denso en $X$. Demuestra también que existen hiperplanos densos.(\textbf{Sugerencia:demuestra que si $f:X\rightarrow \mathbb{R}$ es lineal y no continua, entonces 
\begin{equation*}
f(B_X(x_0;r))=\mathbb{R},\quad \forall x_0\in X,\:\:\forall r>0
\end{equation*}})

Si $(E,\|\cdot\|_E)$ y $(F,\|\cdot\|_F)$ son espacios normados, $\mathcal{L}(E,F)$ va a denotar el conjunto de aplicaciones lineales continuas de $E$ en $F$. En el espacio $\mathcal{L}(E,F)$ se puede definir una norma:
\begin{equation*}
\|f\|=\sup_{\|x\|\leq 1}\|f(x)\|=\inf\{k\geq 0:\|f(x)\|\leq k\|x\|,\:\forall x\in E\}
\end{equation*}

Es fácil comprobar que es una norma en $\mathcal{L}(E,F)$. 

Si $dim(E)=n$, $dim(F)=m$, $\mathcal{L}(E,F)$ no es sino el conjunto de matrices reales $\mathcal{M}_{n\times m}$ y la norma depende de las normas que se elijan en $E$ y en $F$.\\

\textbf{Teorema:} Si $F$ es un espacio de Banach, entonces $\mathcal{L}(E,F)$ es un espacio de Banach con la norma usual de los operadores.\\

\textbf{Demostración:} Sea $(f_p)_{p\in\mathbb{N}}$ una sucesión de Cauchy en $\mathcal{L}(E,F)$. Entonces
\begin{equation*}
\forall \varepsilon \in \mathbb{R}^+\:\:\exists p_0(\varepsilon):p,q\geq p_0(\varepsilon)\Rightarrow \|f_p-f_q\|_{\mathcal{L}(E,F)}\leq \varepsilon
\end{equation*}

lo que implica 
\begin{equation*}
\forall \varepsilon \in\mathbb{R}^+,\:\:\exists p_0(\varepsilon),\:p,q\geq p_0(\varepsilon)\Rightarrow \|f_p(x)-f_q(x)\|\leq \varepsilon \|x\|,\quad \forall x\in E
\end{equation*}

Con lo anterior tenemos:
\begin{itemize}
\item Si $x\in E$ es fijo, entonces $(f_p(x))_{p\in\mathbb{N}}$ es una sucesión de Cauchy en $F$. Como $F$ es un espacio de Banach, entonces $(f_p(x))_{p\in\mathbb{N}}$ converge a un elemento de $F$. Podemos así definir un operador
\begin{equation*}
f:E\rightarrow F,\quad x\mapsto \lim_{p\to+\infty} f_p(x),\quad \forall x\in E
\end{equation*}

\item $f(\alpha x+\beta y)=\lim_{p\to +\infty} f_p(\alpha x+\beta y)=\lim_{p\to +\infty}(\alpha f_p(x)+\beta f_p(y))=\alpha f(x)+\beta f(y),\quad \forall x,y\in E,\:\forall \alpha,\beta\in \mathbb{R}$

\item Lo anterior significa que $f$ es lineal. Veamos que es continuo:

Fijamos $x\in E$, $\varepsilon >0$, $p\geq p_0(\varepsilon)$, $p\to +\infty$ y tenemos
\begin{equation*}
\|f_p(x)-f(x)\|\leq \varepsilon \|x\|,\quad\forall x\in E
\end{equation*}

Así
\begin{equation*}
\|f(x)\|=\|f(x)-f_p(x)\|+\|f_p(x)\|\leq (\varepsilon +\|f_p\|)\|x\|,\quad\forall x\in F
\end{equation*}

Por tanto, $f$ es continuo

\item Finalmente, veamos que $(f_p)_{p\in \mathbb{N}}\xrightarrow{\mathcal{L}(E,F)}f$.

Obtenemos que $\|(f_p-f)(x)\|\leq \epsilon$, $\forall x\in E:\|x\|\leq 1$, luego:
\begin{equation*}
\forall \varepsilon>0,\:\exists p_0(\varepsilon):p\geq p_0(\varepsilon)\Rightarrow \|f_p-f\|_{\mathcal{L}(E,F)}\leq \varepsilon,\quad c.p.d
\end{equation*}

En particular, obtenemos el siguiente resultado notable:\\

Si $(E,\|\cdot\|_E)$ es cualquier espacio normado, el espacio normado $\mathcal{L}(E,\mathbb{R})$, con la norma $\|f\|=\sup_{\|x\|\leq 1}|f(x)|\quad \forall f\in \mathcal{L}(E,F)$ es un espacio de Banach, al que denotaremos $E'$ ($E^*$ en algunos textos) y al que denotaremos dual topológico de $E$.\\

\textbf{Teorema (F.Riesz, 1918):} Sea $(E,\|\cdot\|)$ normado. Entonces $dim(E)$ es finita $\Leftrightarrow$ $\overline{B}_E(0;1)$ es compacto.\\

\textbf{Demostración:} $\Rightarrow)$ Tenemos que la bola cerrada unidad es cerrada y acotada. Ahora bien, como la $dim(E)$ es finita, tenemos que $\overline{B}_E(0;1)$ es compacto (es un corolario del teorema de Hausdorf).

$\Leftarrow)$ Sea $B=\overline{B}_E(0;1)$ que, por hipótesis, es compacto.  Si $\varepsilon \in (0,1)$ es fijo, el conjunto
\begin{equation*}
\{B_E(b;\varepsilon),\:b\in B\}
\end{equation*} 

es un recubrimiento por abiertos de $B$. Por ser $B$ compacto, existe algún subrecubrimiento finito $F=\{b_1,b_2,\ldots,b_n\}\subset B$ tal que
\begin{equation*}
B\subset \bigcup_{i=1}^n B_E(b_i,\varepsilon)=\bigcup_{i=1}^n(b_i+\varepsilon B_E(0;1))
\end{equation*}

Sea $M$ el subespacio de $E$ generado por el subrecubrimiento finito. Entonces deducimos
\begin{equation*}
B\subset M+\varepsilon B
\end{equation*}

Luego
\begin{equation*}
B\subset M+\varepsilon B\subset M+\varepsilon(M+\varepsilon B)=M+\varepsilon M+\varepsilon^2 B=M+\varepsilon^2 B
\end{equation*}

es "fácil" darse cuenta que
\begin{equation*}
B\subset M+\varepsilon^n B,\quad \forall n\in\mathbb{N}
\end{equation*}

Esto implica que $B\subset M$. En efecto, se obtiene:
\begin{equation*}
\forall b\in B,\:\exists m\in M,\:b'\in B:\quad b=m_n+\varepsilon^nb'
\end{equation*}
luego $\|b-m_n\|\leq \varepsilon^n$, $\forall n\in \mathbb{N}$. Como $\varepsilon^n\to 0$, $m_n\to b$. Así $b\in \overline{M}=M$.

En conclusión $B\subset M$. Pero $B$ es la bola cerrada unidad de $E$ y $M$ un suberpacio de $E$, luego $E=M$. En efecto
\begin{equation*}
\forall x\in E\backslash\{0\},\:\frac{x}{\|x\|}\in B\subset M\Rightarrow \frac{x}{\|x\|}\in M\Rightarrow x\in M
\end{equation*}

Finalmente como $M$ tiene dimensión finita, $E$ es de dimensión finita.
\end{itemize}

\end{document}
