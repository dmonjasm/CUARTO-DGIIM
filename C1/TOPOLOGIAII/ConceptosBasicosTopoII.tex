\documentclass{article}
\author{Daniel Monjas Miguélez}

\title{Topología II: Conceptos Básicos}

\usepackage[spanish]{babel}
\usepackage[utf8]{inputenc}
\usepackage{hyperref}
\usepackage{amsmath}
\usepackage{amssymb}

\begin{document}
\maketitle
\newpage
\tableofcontents
\newpage

\section{Grupo Fundamental}
\textbf{Definición:} Sea $X$ un espacio topológico. Un lazo en $X$ con base un punto del espacio, $x\in X$ es un arco $\alpha:[0,1]\rightarrow X$ continuo con $\alpha(0)=\alpha(1)=x$. Se denota $\Omega_{x}(X)$ al conjunto de todos los lazos en $X$ con base $x$. \\

Sean $\alpha,\:\beta\in \Omega_{x}(X)$, se define el producto de lazo como
\begin{gather*}
\alpha*\beta:[0,1]\rightarrow X \\
(\alpha*\beta)(t)=\left\lbrace \begin{array}{c}
\alpha(2t)\quad si\:0\leq t\leq \frac{1}{2} \\
\beta(2t-1)\quad si\:\frac{1}{2}\leq t\leq 1
\end{array} \right.
\end{gather*}

\textbf{Definción:} Sean $\alpha,\:\beta\in \Omega_x(X)$, se dicen que son homotópicos, y se denota por $\alpha\simeq \beta$, si existe una aplicación:
\begin{equation*}
H:[0,1]\times[0,1]\rightarrow X\quad continua\:y:
\end{equation*}

\begin{itemize}
\item $H(t,0)=\alpha(t)\quad \forall t\in [0,1]$, es decir, $H(*,0)=\alpha$.

\item $H(t,1)=\beta(1)\quad \forall t\in [0,1]$, es decir, $H(*,1)=\beta$.

\item $H(0,s)=H(1,s)=x\quad \forall s\in [0,1]$, es decir, $H(0,*)=H(1,*)=\varepsilon_x$
\end{itemize}

Se dice que $H$ es un homotopía de $\alpha$ a $\beta$, y se escribe:
\begin{equation*}
H:\alpha\simeq \beta
\end{equation*}

\textbf{Propiedades de las homotopías:}
\begin{enumerate}
\item Si $\alpha\in \Omega_x(X)$, entonces $\alpha\simeq \alpha$ con $H:[0,1]\times [0,1]\rightarrow X$ tal que $H(t,s)=~\alpha(t)$.

\item Si $h:[0,1]\rightarrow [0,1]$ es un homomorfismo con $h(0)=0$ y $h(1)=1$ entonces $\alpha\simeq \alpha\circ h$ donde $\alpha \circ h$ es un reparametrización de $\alpha$ preservando orientación.

\item Sea $\alpha,\beta \in \Omega_x(X)$. Si $\alpha\simeq \beta$ entonces $\beta \simeq \alpha$.

\item Sean $\alpha,\:\beta\in \Omega_x(X)$. Si $\alpha\simeq \beta$ y $\beta\simeq \gamma$ entonces $\alpha\simeq \gamma$.
\end{enumerate}

\textbf{Proposición:} Sean $X$ un espacio topológicos y puntos $p,q,r\in X$. Sean $\alpha,\alpha'\in \Omega_{p,q}(X)$ y $\beta,\beta'\in \Omega_{q,r}(X)$ arcos tales que $\alpha\simeq \alpha'$ y $\beta\simeq \beta'$. Entonces $\alpha*\beta\simeq \alpha'*\beta'$.

\textbf{Proposición:} Sean $X$ un espacio topológico y puntos $p,q,r,s\in X$. Sean $\alpha \in \Omega_{p,q}(X)$, $\beta\in \Omega_{q,r}(X)$ y $\gamma\in \Omega_{r,s}(X)$. Las siguientes propiedades son ciertas:

\begin{itemize}
\item $\alpha*(\beta*\gamma)=(\alpha*\beta)*\gamma)$

\item $(\alpha*\varepsilon_p=\varepsilon_p*\alpha=\alpha$

\item $	\alpha*\overline{\alpha} = \varepsilon_p$
\end{itemize}

\textbf{Teorema:} Sea $X$ un espacio topológico y $p\in X$ un punto arbitrario. La ley de composición interna
\begin{equation*}
*:\Pi_1(X,p)\times\Pi_1(X,p)\rightarrow\Pi_1(X,p)\qquad [\alpha]*[\beta]=[\alpha*\beta]
\end{equation*}

está bien definida y dota al conjunto $\Pi_1(X,p)$ de estructura de grupo algebraico.

El grupo $(\Pi_1(X,p),*)$ es conocido como \textbf{Grupo Fundamental o de Poincaré} del espacio en el punto $p$. Recalcar que $\Pi_1(X,p)=\Omega_p(X)/\simeq$.\\

\textbf{Proposición:} Sea $(X,\tau)$ un espacio arcoconexo, $x,y\in X$. Entonces los grupos $\Pi_1(X,x)$ y $\Pi_1(X,y)$ son isomorfos.\\

\textbf{Observación:} Sea $\gamma$ un arco que une los puntos $x_1,x_2\in X$ entonces 
\begin{equation*}
\phi:\Pi_1(X,x_1)\rightarrow \Pi_1(X,y),\qquad \phi([\alpha])=[\gamma^{-1}][\alpha][\gamma]
\end{equation*}

es un isomorfismo de grupos. 

\textbf{Corolario:} El grupo fundamental $\Pi_1(X,p)$ está unívocamente determinado salvo isomorfismos por la arcocomponente $C_p$ del punto $p$. En particular, si $X$ es arcoconexo entonces la clase de isomorfía de $\Pi_1(X,p)$ no depende del punto $p\in X$. En este caso la notación es $\Pi_1(X)$. \\

\textbf{Proposición:} Sean $X$ e $Y$ espacios topológicos y $\varphi:X\rightarrow Y$ una aplicación continua. Consideremos $\alpha,\beta\in \Omega_{p,q}(X)$ y los correspondientes $\varphi\circ\alpha,\varphi\circ\beta\in \Omega_{\varphi(p),\varphi(q)}(Y)$. Se tiene que

\begin{equation*}
\alpha\simeq \beta \Rightarrow \varphi\circ \alpha \simeq \varphi\circ\beta
\end{equation*}

En particular:
\begin{itemize}
\item La aplicación $\varphi_*:\Pi_1(X,p)\rightarrow \Pi_1(Y,\varphi(p)),\quad\varphi_*([\alpha])=[\varphi\circ\alpha]$ está bien definida yes un homomorfismo de grupos.

\item Si $\psi:Y\rightarrow Z$ es otra aplicación continua y consideramos los homomorfismos de grupos 
\begin{gather*}
\psi_*:\Pi_1(X,\varphi(p))\rightarrow \Pi_1(Y,\psi(\varphi(p)))\\(\psi\circ\varphi)_*~:~\Pi_1(X,p)~\rightarrow~\Pi_1(Z,\psi(\varphi(p)))
\end{gather*}
entonces se tiene que $(\psi\circ\varphi)_*=\psi_*\circ\varphi_*$
\end{itemize}





\end{document}