
\documentclass{article}
\author{Daniel Monjas Miguélez}

\title{Topología II: Ejercicios Resueltos}

\usepackage[spanish]{babel}
\usepackage[utf8]{inputenc}
\usepackage{hyperref}
\usepackage{amsmath}
\usepackage{amssymb}
\usepackage{ textcomp }
\usepackage{ graphicx }
\graphicspath{ {images/} }

\begin{document}
\maketitle

\newpage

\tableofcontents

\newpage

\section{Relación 1}
\subsection{Ejercicio 1}
Sea $I=[0,1]$, y sean $\alpha$ un arco y $\theta$ una aplicación continua tal que $\phi(0)=0$ y $\phi(1)=1$.

Sea 
\begin{equation*}
\left.\begin{array}{c}
H:I\times I\rightarrow X\\
H(t,s)=\alpha((1-s)t+s\phi(t))
\end{array} \right.
\end{equation*}

Ya que $\phi(t)\in I$, el número $(1-s)t+s\phi(t)$ pertenece al intervalo $I$, luego $H$ está bien definida. Por otra parte, dicha aplicación es continua y satisface:
\begin{itemize}
\item $H(t,0)=\alpha(t)$
\item $H(t,1)=\alpha(\phi(t))=\alpha\circ\phi(t)$
\item $H(0,s)=\alpha(0)=\alpha(\phi(0))$
\item $H(1,s)=\alpha(1)=\alpha(\phi(1))$ 
\end{itemize}

, luego existe una homotopía entre $\alpha$ y $\alpha\circ \phi$.

\subsection{Ejercicio 2}
Sea $X\subset \mathbb{R}^n$ un subconjunto estrellado de $\mathbb{R}^n$. Sea $x_0$ el centro de $X$ y sea $\alpha\in\Omega_{x_0}(X)$ un lazo con base el centro de $X$. 

Sea
\begin{equation*}
\left.\begin{array}{c}
H:I\times I\rightarrow X \\
H(t,s)=(1-s)x_0+s\alpha(t)
\end{array}\right.
\end{equation*}

Es calaramente continua y está bien definida, pues para cada $t\in I$, $H(t,s)$ es el segmento que una $x_0$ con $\alpha(t)$, el cual está contenido en $X$, pues al ser $X$ estrellado todo segmento desde el centro de $X$ a un punto de $X$ está contenido en $X$.\\

Para la segunda parte, sean $\alpha, \beta\in \Omega_{p,q}(X)$ con $p,q\neq x_0$. Sean 
\begin{equation*}
\left.\begin{array}{c}
e_1:[0,1]\rightarrow X\\
t\mapsto (1-t)x_0+tp
\end{array} \right.\qquad
\left.\begin{array}{c}
e_2:[0,1]\rightarrow X\\
t\mapsto (1-t)q+tx_0
\end{array} \right.
\end{equation*}

Estos dos segmentos están contenidos en $X$, pues son segmentos que unen el centro del conjunto con otros puntos y por ser $X$ estrellado está bien definido. Como $e_1\in \Omega_{x_0,p}(X)$ y $e_2\in \Omega_{q,x_0}(X)$, entonces $e_1*\alpha*e_2,e_1*\beta*e_2\in \Omega_{x_0}(X)$. Como ya hemos visto que todo lazo del centro del conjunto es homotópico al lazo constante se tiene que
\begin{equation*}
e_1*\alpha*e_2\sim \varepsilon_{x_0}\sim e_1*\beta*e_2\Rightarrow \alpha\sim \beta
\end{equation*}

\subsection{Ejercicio 3}
Veamos que 
\begin{gather*}
\phi_\gamma =\phi_\sigma\Leftrightarrow [\overline{\gamma}*\alpha*\gamma]=[\overline{\sigma}*\alpha*\sigma]\quad \forall [\alpha]\in \Pi_1(X,y) \Leftrightarrow\\
\Leftrightarrow [\overline{\gamma}]*[\alpha]*[\gamma]=[\overline{\sigma}]*[\alpha]*[\sigma] \quad \forall [\alpha] \in \Pi_1(X,y)\Leftrightarrow \\
\Leftrightarrow [\alpha]*[\gamma]*[\sigma]^{-1}=[\overline{\gamma}]^{-1}*[\overline{\sigma}]*[\alpha] \quad \forall [\alpha]\in \Pi_1(X,y)
\end{gather*}
, basta con usar que $[\sigma]^{-1}=[\overline{\sigma}]$. Esto es claro, pues $[\sigma]^{-1}*[\sigma]=[\epsilon_y]=[\overline{\sigma}*\sigma]=[\overline{\sigma}]*[\sigma]$.

De aquí se obtiene pues $[\alpha]*[\gamma]*[\overline{\sigma}]=[\gamma]*[\overline{\sigma}]*[\alpha] \quad \forall [\alpha]\in \Pi_1(X,y)$. Pero de aquí se deduce claramente que $[\gamma*\overline{\sigma}]\in \mathcal{Z}(\Pi_1(X,y))$

\subsection{Ejercicio 4}
Basta con ver que el cono de un espacio $CX$ es $X\times I/\sim$, donde $(x,1)\sim (y,1)$, $\forall (x,y)\in X$.

Observemos que cada $[(x,t)]$ de $CX$ puede conectarse al vértice $[(x,1)]=X\times \{1\}$, mediante el segmento $[(x,(1-s)t+s)]$, $t\in [0,1]$, que esta contenido en $CX$.

Como todo punto de $CX$ se puede unir con $[(x,1)$ entonces se tiene que el cono es un espacio arcoconexo.

\subsection{Ejercicio 5}
$\mathbb{S}^1$ es un retracto fuerte de deformación de $\mathbb{R}^2\backslash \{(0,0)\}$, basta con definir $r(x,y)=(x,y)/\sqrt{x^2+y^2}$. Por tanto $\alpha$ es homotópica a $r\circ \alpha$.

Pero es claro que $r\circ \alpha=\alpha$, luego no es homotópica al lazo constante, en particular la clase de homotopía de $\alpha$ es un generador del grupo fundamental de $\mathbb{R}^2\backslash\{(0,0)\}$ que es $\mathbb{Z}$.

\subsection{Ejercicio 6}
Supongamos que $\mathbb{R}\times [0,\infty)$ es homeomorfo a $\mathbb{R}^2$, entonces 
\begin{equation*}
\mathbb{R}\times [0,\infty)\backslash \{(0,0)\}\cong\mathbb{R}^2\backslash\{f(0,0)\}
\end{equation*}

, pero esto no es posible, pues el grupo fundamental de $\mathbb{R}^2\backslash\{f(0,0)\}$ es $\mathbb{Z}$, mientras que $\mathbb{R}\times [0,\infty)\backslash \{(0,0)\}$, sigue siendo simplemente conexo, luego no pueden ser homeomorfos.

\subsection{Ejercicio 7}
El único caso qeu se va a estudiar es el (f). Los demás se obtiene rápidamente por ser estrellados, o usando la proposición 8.1 de los apuntes de jlopez.

En cuanto al caso $(f)$ basta ver que $f:\mathbb{R}^2\rightarrow \mathbb{R}$, tal que $(x,y)\mapsto z=x^2-y^2$ es una apliación continua. Basta con tomar que el grafo de esta aplicación $G(f)$ que coincide con el paraboloide hiperbólico es homeomorfo al dominio de $f$, el cual sabemos que es simplemente conexo.

\subsection{Ejercicio 8}

\subsection{Ejercicio 9}
$\mathbb{T}=\left\langle (x,y,z)\in \mathbb{R}^3| x=cos\theta\cdot(R+rcos\varphi),\:y=sen\theta\cdot(R+rcos\varphi), z=rsin\varphi\right\rangle$, donde $R$ es la distancia entre el centro del donut y el centro de uno del conducto y $r$ es el radio del conducto.
\begin{figure}[h]
\centering
\includegraphics[scale=0.5,width=\textwidth]{toro-volumen.jpg}
\end{figure}

\subsection{Ejercicio 12}
\begin{itemize}
\item Si $f$ es inyectiva, también lo es $f_*$.

Esto no es cierto. Basta tomar la aplicación inclusión como sigue
\begin{gather*}
i:\mathbb{S}^1\rightarrow \mathbb{R}^2\qquad i_*:\Pi_1(\mathbb{S}^1,x)=\mathbb{Z}\rightarrow \Pi_1(\mathbb{R}^2,x)=\{[\varepsilon_x]\}
\end{gather*}
, donde claramente $i_*$ no es una aplicación inyectiva.

\item Si $f$ es sobreyectiva, también lo es $f_*$.
\begin{gather*}
f:\mathbb{R}\rightarrow \mathbb{S}^1\qquad f(t)=(\cos2\pi t,\sen2\pi t)\\
f_*:\{[\epsilon_p]\}\rightarrow \mathbb{Z}
\end{gather*}

y esta aplicación claramente no es sobreyectiva.

\item Si $f$ es biyectiva, también lo es $f_*$.
\begin{gather*}
f:[0,1[\rightarrow \mathbb{S}^1\qquad f(t)=(\cos(2\pi t),\sen(2\pi t))\\
f_*:\{[\varepsilon_p]\}\rightarrow \mathbb{Z}
\end{gather*}

, donde $f$ es claramente continua, inyectiva y sobreyectiva y $f_*$ no puede serlo.
\end{itemize}

\subsection{Ejercicio 13}
\begin{itemize}
\item Claramente $\emptyset$ y $X$ están incluidos en la topología.

\item Por otro lado los abiertos son los conjuntos infinitos, $X$ y $\emptyset$. Claramente la unión de conjuntos abiertos (infinitos) es un conjunto abierto (infinito).

\item Por otro lado supongamos que la intersección de conjuntos abiertos no fuese un conjunto abierto (infinito). Entonces $X\backslash O_1\cap O_2$ sería un conjunto infinito. Pero se tiene que 
\begin{equation*}
X\backslash O_1\cap O_2 = X\backslash O_1\cup X\backslash O_2
\end{equation*}

que es la unión de dos conjuntos finitos, pues sus complementarios son cerrados (finitos) y por tanto la unión de finitos es finita, en contra de la suposición de que el complementario de la intersección no era un conjunto cerrado.
\end{itemize}

Luego hemos deducido que efectivamente, es una topología.\\

Para ver que es arcoconexo, basta tomar dos puntos cuales quiera de $X$ y que por como se ha definido $X$ entre dos puntos cualesquiera habrá exactamente $k$ números naturales distintos de los puntos. Basta con tomar como arco una aplicación tal que 
\begin{gather*}
\alpha:[0,1]\rightarrow X\\
\alpha(t)=\left\lbrace\begin{array}{c}
\alpha(0)=x\\
\alpha(1)=y\\
\alpha(t)=i \quad t=\frac{i}{k+1}\\
\alpha(t)=\infty \quad t\neq \frac{i}{k+1}
\end{array}\right.
\end{gather*}

esta aplicación es continua, pues la preimagen de abiertos es abierto y la preimagen de cerrados es cerrada. Y constituye un arco entre cualesquiera dos puntos de $X$, luego se tiene que $X$ es arcoconexo.

Por otro lado es trivial que es simplemente conexo, pues este espacio es contráctil, y por tanto simplemente conexo.

\subsection{Ejercicio 16}
Supongamos que $\exists\gamma \in \Omega_a(X)$, tal que $\gamma\notin \Omega_a(A)$. Veamos que esto no es posible. 

Si este arco existiera, se tendría que existen valores de $t$ con $t\in I$ tal que $\gamma(t)\notin A$. Pero entonces existe el arco que une $a\in A$ con dichos $\gamma(t)$. De aquí se deduciría que $A\cup \gamma(t)$ sigue siendo arcoconexo, en contra de que $A$ es componente arcoconexa. Por tanto se verifica que $\forall \gamma \in \Omega_a(X)$ se verifica que $\gamma \in \Omega_a(A)$. 

Una vez demostrado esto tendremos que si $\gamma\in \Omega_a(X)$, entonces
\begin{equation}
F_\gamma:\Pi_1(X,a)\rightarrow \Pi_1(X,a)
\end{equation}

es un isomorfismo, pero como todo lazo de $\Omega_a(X)$ está en $\Omega_a(A)$ entonces $\Pi_1(X,a)=\Pi_1(A,a)$ luego por el isomorfismo anterior se tiene el resultado.

\subsection{Ejercicio 17 (Ejercicio 1 Retracciones)}
En primer lugar veamos que es claro que $\mathbb{S}^1\cong I\times \{1/2\}/\sim$.

Con esto claro, vamos a ver que $I\times \{1/2\}/\sim$ es un retracto de deformación de $\mathbb{M}$ y, como es isomorfo a $\mathbb{S}^1$ también lo es $\mathbb{S}^1$. En primer lugar vamos a definir la siguiente aplicación, y veamos que es una retracción:
\begin{gather*}
r:I\times I/\sim \rightarrow I\times \{1/2\}/\sim\\
[(x,y)]\mapsto [(x,1/2)]
\end{gather*}

Esta aplicación está bien definida, pues si $(x,y)$ es un punto que no está en el borde $\{0,1\}\times I$, se relaciona únicamente consigo mismo y por consiguiente la aplicación está definida.

Si el punto $(x,y)$ es del borde anteriormene mencionado, veamos que independientemente de que se cambie el representante de la clase sigue estando bien definido. Esto es cierto, pues si cogemos un representante del borde al aplicarle la imagen nos queda que su imagen es una clase de equivalencia con dos elementos, la imagen del representante y la imagen del putno con el que se relaciona el representante y por consiguiente está bien definido.

En cuanto a la continuidad se puede ver con facilidad que $I\times I \xrightarrow{f} I\times \{1/2\}\xrightarrow{p}I\times\{1/2\}/\sim$ es una aplicación continua, pues al componer $f$ con las proyecciones $p_1$ y $p_2$ nos queda la $Id_I$ y una aplicación constante, luego $f$ es constante y al componer $f$ con la proyección en el cociente es una aplicación continua, luego $r$ es continua.\\

Para la deformación tomamos
\begin{gather*}
F:\mathbb{M}\times [0,1]\rightarrow \mathbb{M}\\
([(x,y)],t)\mapsto [(1-t)(x,y)+tr(x,y)]
\end{gather*}
, es aplicación está bien definida, pues antes de aplicar el cociente claramente el segmento que une dos puntos de $I\times I$ está contenido en $I\times I$ y por tanto la proyección de cualquiera de esos segmentos está en la banda de Möbius.

Por otro lado para ver que es continua, basta ver que en este caso como
\begin{gather*}
p\times 1_{[0,1]}\\
((x,y),t)\mapsto [(1-t)(x,y)+tr(x,y)]
\end{gather*}

es una aplicación continua por ser el producto cartesiano de aplicaciones continuas, entonces nuestra deformación es continua, y por consiguiente $I\times\{1/2\}/\sim$ es retracto de deformación de $\mathbb{M}$ y por tanto $\mathbb{S}^1$ es retracto de deformación.

\subsection{Ejercicio 18 (Ejercicio 2 Retracciones)}
La curva borde de este espacio es $I\times \{0,1\}$, y para ver que es homeomorfa a $\mathbb{S}^1$, basta tomar un arco $\alpha$ que recorra el borde y al hacer el cociente quedará que cada punto del borde se relacione únicamente consigo mismo, menos los puntos del borde que en $x$ valen $0$ o $1$, que se relacionan con otro.

Por otro lado para ver que valor tiene en el grupo fundamental basta usar la aplicación:
\begin{gather*}
r_*:\Pi_1(X)\rightarrow \Pi_1(A)\\
r_*([\alpha])=[r\circ \alpha]
\end{gather*}

y el valor de esta composición será el número de vueltas que se de a la banda de Möbius al recorrer el arco $\alpha$, que al hacer el recorrido se verá que da 2 vueltas exactas.

\subsection{Ejercicio 19 (Ejercicio 3 Retracciones)}
Este ejercicio lo resolveré usando resultado de análisis funcional. Basta con utilizar el teorema de mejor aproximación, ya que se cumplen las hipótesis de que $\mathbb{R}^n$ con el producto usual es un espacio de Hilbert, y por el enunciado $A\subset \mathbb{R}^n$ es compacto (cerrado y acotado) y convexo. De este teorema se obtiene que $\forall x\in \mathbb{R}^n$ existe en $A$ un único $a_x$ tal que $d(x,a_x)=d(x,A)$. 

Además la aplicación proyección 
\begin{gather*}
P_A:\mathbb{R}^n\rightarrow A\\
x\mapsto a_x
\end{gather*}

es lipschitziana y por tanto continua. Con esto visto veamos que $P_A$ en sí misma es una retracción.\\

Ya tenemos que es continua, únicamente nos falta ver que para cada $x\in A$, $P_A(x)=x$, pero esto es trivial, pues si $x\in A$, entonces $d(x,A)=d(x,a_x)=0\Rightarrow a_x=x$ y por consiguiente al restringirlo a $A$ $P_A$ es la identidad.

\subsection{Ejercicio 20 (Ejercicio 4 Retracciones)}
Supongamos $(0,0)\in U\subset\mathbb{R}\times [0,\infty)$ un entonrno. Supongamos $U\cong \mathbb{R}^2$. Al ser $U$ un entorno es de la forma $U'\cap \mathbb{R}\times [0,\infty)$ con $U'$ entorno del $(0,0)$ en $\mathbb{R}^2 \Rightarrow \exists (0,0)\neq (x,0)\in U\Rightarrow F_{|U\backslash\{(x,0)\}}:U\backslash\{(x,0)\}\cong \mathbb{R}^2\backslash\{F(x,0)\}$. Pero es fácil ver que el primero es simplemente conexo, mientras que el segundo es homeomorfo a $\mathbb{S}^1$ y por tanto su grupo fundamental es $\mathbb{Z}$.

\subsection{Ejercicio 21 (Ejercicio 5 Retracciones)}
Tomaremos la siguiente retracción
\begin{gather*}
r:\mathbb{B}_s\rightarrow \mathbb{D}_r\\
(x,y)\mapsto r(x,y)=\left\lbrace\begin{array}{c}
(x,y)\quad (x,y)\in int(\mathbb{D}_r)\\
\frac{(x,y)}{\|(x,y)\|} \quad (x,y)\in \mathbb{B}_s\backslash int(\mathbb{D}_r)
\end{array}\right.
\end{gather*}

, la cual está bien definida, pues los puntos del disco los deja donde están y los puntos de fuera del disco los lleva a la frontera del disco. Además es claramente continua, pues lleva cerrados en cerrados y abiertos en abiertos.

Para la deformación tomaremos
\begin{gather*}
H:\mathbb{B}_s\times I\rightarrow \mathbb{B}_s\\
H(x,t):=(1-t)x+tr(t)
\end{gather*}

, esta aplicación está bien definida, pues fijado $x\in \mathbb{B}_s$ se verifica que $H$ es el segmento que une $x$ con la imagen de su retracción. Como $\mathbb{B}_s$ es convexo es claro que todos estos segmentos están completamente contenidos en $\mathbb{B}_s$ y por tanto está bien definida. Además es claramente continua. 

De aquí se deduce que es un retracto de deformación.

\subsection{Ejercicio 22 (Ejercicio 6 Retracciones)}

\subsection{Ejercicio 23 (Ejercicio 7 Retracciones)}
Sea $X$ un espacio topológico $A\subset X$ un retracto de deformación de $X$. ¿Se verifica que si $Y$ es un espacio homeomorfo a $X$ entonces $F(A)\subset Y$ es retracto de deformación?\\

En primer lugar como $A$ es un retracto de deformación entonces existe
\begin{gather*}
r:X\rightarrow A\\
\end{gather*}

es una retracción, entonces
\begin{gather*}
f \circ r \circ f^{-1}:Y\rightarrow f(A)
\end{gather*}

es una aplicación continua, pues es composición de continuas y además veamos que al restringirlo a los valores de $f(A)$ es la identidad. Basta tomar un punto $y\in f(A)$, por ser $f$ un homeomorfismo $f^{-1}(y)\in A$, al aplicarle la retracción $r(f^{-1}(y)) = f^{-1}(y)$ y por último al volver a aplicar $f$ se vuelve a $y$, lo que implica que al restrigir esa aplicación a los valores de $f(A)$ es la identidad en $f(A)$ y por tanto es una retracción. 

Para la deformación seguimos un procedimiento similar
\begin{gather*}
H'\equiv f\circ H:X\times [0,1]\rightarrow X\\
H'(y,t)=H(f^{-1}(y),t)
\end{gather*}

es una aplicación continua, y está bien definida. Además cumple las propiedades de una deformación.

De aquí se deduce que los retractos de deformación se mantienen por homeomorfismos.

\subsection{Ejercicio 24 (Ejercicio 8 Retracciones)}
Si $A\subset X$ es retracto de deformación de $X$ y $B\subset A$ es retracto de deformación de $A$
\begin{gather*}
\left.\begin{array}{c}
r:X\rightarrow A\\
r':A\rightarrow B
\end{array}\right\rbrace r'\circ r:X\rightarrow B
\end{gather*}

es una retracción. Es continua, pues es composición de continua y claramente restringido a valores de $B$ es la identidad.

Para la deformación sea
\begin{gather*}
H':X\times [0,1]\rightarrow X\\
H'(q,t):=\left\lbrace \begin{array}{c}
H(q,2t)\quad (q,t)\in X\times[0,1/2]\\
H''(r(q),2t-1)\quad t\in [1/2,1]
\end{array}\right.
\end{gather*}

y esta aplicación es claramente continua, pues para cada $q$ es el producto de arcos que unen  respectivamente $q$ con $r(q)$ y $r(q)$ con $r'(q)$ por medio de un arco.

\subsection{Ejercicio 25 (Ejercicio 9 Retracciones)}
Basta con tomar las siguientes retracciones y deformaciones
\begin{gather*}
r:X\rightarrow A\\
r':Y\rightarrow B\\
r\times r':X\times Y \rightarrow A\times B
\end{gather*}

es una aplicación continua, pues al componerlo con las proyecciones nos quedan las retracciones iniciales y está bien definida y al restringirloa a $A\times B$ es la identidad por como se ha definido.

Para la deformación consideraremos lo siguiente
\begin{gather*}
H:X\times [0,1]\rightarrow X\\
H':Y\times [0,1]\rightarrow Y\\
H'':(X\times Y)\times [0,1]\rightarrow X\times Y\\
H''((x,y),t):=(H(x,t),H'(y,t))
\end{gather*}

que de nuevo es continua pues al componerlo con la proyecciones nos quedan las deformaciones previas y por tanto es continua. Además se verifica que si $t=0$ se da la identidad y que si $t=1$ se da el producto cartesiano de las retracciones.

\subsection{Ejercicio 26 (Ejercicio 10 Retracciones)}
Como $A\cap B$ es retracto de $B$ entonces existen
\begin{gather*}
r:B\rightarrow A\cap B\\
H:B\times [0,1]\rightarrow B
\end{gather*}

son respectivamente una retracción y una deformación.

Definimos la siguiente aplicación
\begin{gather*}
s(x)=\left\lbrace \begin{array}{c}
x\qquad x\in A\\
r(x)\qquad x\in B
\end{array}\right.
\end{gather*}

es claro que está bien definida y como está definida en dos abiertos (resp. dos cerrados) es continua en cada uno de ellos y por tanto r es continua. Por otro lado es trivial que al restringirlo $A$ es la identidad.

Por otro lado definimos 
\begin{gather*}
G:X\times [0,1]\rightarrow X\\
G(x,t)=\left\lbrace\begin{array}{c}
x\qquad x\in A\\
H(x,t) \qquad x\in B
\end{array}\right.
\end{gather*}

la cual es continua por lo dicho antes. Si $t=0$ es claro que se trata de la identidad para todo $x$ y si $t=1$ es claro que es $s(x)$, luego es una deformación y por consiguiente hemos llegadoa a que $A$ es un retracto de deformación de $X$.

\subsection{Ejercicio 27 (Ejercicio 11 Retracciones)}
Un espacio es Hausdorff o $T_2$ si y sólo si el espacio diagonal es cerrado, es decir, si
\begin{equation*}
\Delta:=\{(x,x)\in X\times X\} 
\end{equation*}

es cerrado.

Definimos la siguiente aplicación 
\begin{gather*}
f:X\rightarrow X\times X\\
x\mapsto (x,r(x))
\end{gather*}

Esta aplicación es continua pues al componerla con las proyecciones se obtiene la aplicación identidad y la retracción de $X$ en $A$, siendo ambas continuas. Por otro lado veamos que claramente
\begin{equation*}
f^{-1}(\Delta)=\{x\in X: x=r(x)\}=A
\end{equation*}

, luego por ser $f$ continua $A$ es cerrado.

\subsection{Ejercicio 28 (Ejercicio 15 Retracciones)}
Basta ver que si un espacio es contráctil, entonces existe una deformación tal que
\begin{equation*}
H:X\times [0,1]\rightarrow X
\end{equation*}

y fijando cada $x\in X$ se tiene que $H_x(t)$ es un arco que una $x$ con $r(x)$, pero como es contráctil $r(x)=\{p\}$. Por consiguiente para cada punto $x\in X$, $H_x(t)$ constituye un arco entre $x$ y $p$ y por tanto el espacio es arcoconexo.

\section{Relación 2}
$(\tilde{X},p)$ es un espacio recubridor de $X$.

\subsection{Ejercicio 1}
Como $p:\tilde{X}\rightarrow X$ es una aplicación recubridora entonces es un homeomorfismo local y sea $\tilde{x}\in \tilde{X}$ y $W \in U_{\tilde{x}}$ un entonrno de esta. Veamos que $p(W)$ es un entorno de $p(\tilde{x})$. Ya que $p$ es un homemomorfismo local existen $U\in U_{\tilde{x}}$ y $U'\in U_{p(\tilde{x})}$ tal que $p:U\rightarrow U'$ es un homemomorfismo. Entonces $U\cap W$ es un entorno de $\tilde{x}$ en la topología relativa de $U$ y como $p$ es homeomorfismo entre $U$ y $U'$, el conjunto  $p(U\cap W)$ es un entorno de $p(\tilde{x})$ en $X$. Por tanto $p(U\cap W)\subset p(W)$ y por tatno $p(W)$ es un entorno de $p(\tilde{x})$. Como la apliación lleva entornos en entornos la aplicación es abierta.

\subsection{Ejercicio 2}
Sea $q\in X$ y $O\subset X$ con $q\in O$. $\forall U$ entorno distinguido alrededor de $q$ la arcocomponente del abierto $U\cap O$ que contiene a $q$ satisface $q\in V\subset U\cap O\subset O$ y pertenecece a la familia de abiertos distinguidos. De aquí se deduce que $\forall O\in \tau$ existe $V\in U$ con $U\subset O\Rightarrow $ la familia de entornos distinguidos es una base.

\subsection{Ejercicio 3}
Sabemos que todo homeomorfismo es un aplicación recubridora. Por un lado veamos que
\begin{gather*}
f\circ p: \tilde{X}\rightarrow Y\\
p\circ g:Z\rightarrow X
\end{gather*}

Para ver que $\tilde{X}$ es recubridor de $Y$ veamos que $\forall U\in X$ entorno distinguido $f(X)$ es entorno distinguido. Claramente $f\circ p$ es continua y sobreyectiva por ser composición de funciones continuas y sobreyectivas. Para los entornos distinguidos usaremos que $f$ es una biyección entre $X$ y $Y$ y que si $U_x$ es un entorno distinguido de un punto $x\in X$ tenemos que $f(U_x)$ es un entorno de $f(x)$ en $Y$ y además es arcoconexo y abierto, pues $f$ es la inversa de $f^{-1}$ que es continua, lo que implica que $f(U_x)$ es abierto. Y para ver que es arcoconexo basta con ver que la imagen de un arcoconexo por una función continua es arcoconexo, luego es abierto y arcoconexo. 

Por último vemos que $f(U_x)$ es entorno distinguido simplemente con ver que como $f$ y $f^{-1}$ es biyectivo $f^{-1}(f(U_x))=U_x$ es entorno distinguido. Y por tanto si $A\subset p^{-1}(f^{-1}(f(U(x)))$ es una arcocomponente entonces es claro que $f\circ p_{|A}$ es un homeomorfismo, pues $p_{|A}$ es un homeomorfismo local, que compuesto con un homoemorfismo sigue siendo un homeomorfismo.\\

Por otro lado de nuevo, $p\circ g$ es una aplicación continua y sobreyectivo. Y si tomamos un entorno distinguido cualquiera de un punto de $X$, veamos que este también es entorno distinguido para esta aplicación. Basta con ver que que $g^{-1}(p^{-1}(U))$ es un conjunto tal que para toda arcocomponente de este su restricción es un homemomorfismo, pues la restricción de un homeomorfismo a cualquier subconjunto sigue siendo un homemorfismo. Y es claro que la imagen de una arcocomponente sigue siendo una arcocomponente y claramente será una arcocomponente de $p^{-1}(U)$ y claramente al restringir $p$ es este conjunto seguirá siendo un homeomorfismo.

\subsection{Ejercicio 4}
Claramente se verifica que $p$ así descrita es una aplicación continua y sobreyectiva por ser una proyección.

Vamos a observar que al ser $Y$ un espacio discreto lo vamos a tomar como unión de puntos aislados. Sea $G\subset Homeo(X\times Y)$ donde $G=\{g_{y_i}: i\in I\}$ tal que cada una de las funciones de $g$ es una traslación (y por tanto homeomorfismo) definida como
\begin{gather*}
g_{y_i}:X\times Y\rightarrow X\times \{y_i\}\\
(x,y)\mapsto (x,y_i)
\end{gather*}

Si $U$ es un entorno abierto y arcoconexo de un punto $p$ entonces es claro que
\begin{gather*}
g\cdot U\cap U=\emptyset \qquad \forall g\in G\backslash \{Id_X\}
\end{gather*}

Por otro lado $X\times Y/G$ es claramente homeomorfo a $X$, pues dos puntos se relaciones si y sólo si existe $g\in G$ tal que $g(x,y)=g(x',y')$, pero por ser $G$ una traslación respecto a $y$ esto es solo posible si $x=x'$. Es decir, cada punto se relaciona únicamente consigo mismo y todos los puntos cuyo valor de $x$ sea el mismo que el del representando, luego hay tantas clases de equivalencia como puntos en $x$ y se pueden relacionar por un homeomorfismo.

Como sabemos que claramente $G$ actúa de forma propia y discontinua entonces la proyección en el espacio de órbitas es una aplicación recubridora y como el espacio cociente es homeomorfo a $X$ entonces tenemos lo buscado que $X\times Y$ recubre a $X$.

\subsection{Ejercicio 5}
$f:Z\rightarrow \tilde{X}$. $f$ es continua por ser una poryección, es más, es continua y sobreyectiva.

Sea $\tilde{x}\in \tilde{X}$, y $x=p(\tilde{x})$. Sea $U$ entorno distinguido de $x$ en $X$ para $(\tilde{X},p)$. Tomamos $\tilde{U}$ tal que $\tilde{x}\in \tilde{U}\subset p^{-1}(U)$. Veamos que $\tilde{U}$ es entorno distinguido de $\tilde{x}$ en $\tilde{X}$ para $(Z,f)$.
\begin{equation*}
f^{-1}(\tilde{U})=\{(\tilde{x}_1,\tilde{x}_2)|\:\tilde{x_1}\in \tilde{U},\quad \tilde{x_2}\in p^{-1}(p(\tilde{x_1})) \} \subseteq \tilde{U}\times p^{-1}(U)
\end{equation*}

Las arcocomponente de $f^{-1}(\tilde{U})$ son de la forma $(\tilde{U}\times \tilde{U}')\cap f^{-1}(\tilde{U})$ con $\tilde{U}'$ arcocomponente de $p^{-1}(U)$. Definimos $\tilde{\tilde{ U}}\equiv (\tilde{U}\times \tilde{U}')\cap f^{-1}(\tilde{U})=\{(\tilde{x}_1,\tilde{x}_2)|\tilde{x}_1\in \tilde{U},\quad \tilde{x}_2\in p^{-1}(p(\tilde{x}_1)) \wedge \tilde{x}_2\in \tilde{U}'\}=\{(\tilde{x}_1,\tilde{x}_2)| x_1\in \tilde{U},\quad x_2\in \tilde{U},\quad p(\tilde{x_1})=p(\tilde{x_2})\}$, donde para cada $x_1\in \tilde{U}$, existe un único $\tilde{x}_2\in \tilde{U}'$ tal que $p(\tilde{x}_1)=p(\tilde{x}_2)$  ya que las arcocomponentes son disjuntas $\Rightarrow f_{|\tilde{\tilde{U}}}:\tilde{\tilde{U}}\cong \tilde{U}$ homeomorfismo.

\subsection{Ejercicio 6}
Sea $\pi_1:Y\rightarrow X$ y $\pi_2:X\rightarrow Z$, donde $\pi_2$ es un recubridor finito.

Sea $z\in Z$ un punto arbitrario y consideremos un entorno distinguido de $U$ en $Z$ para $(X,\pi_2)$ alrededor de $z$. Por ser $\pi_2$ finito se tiene que $\pi_2^{-1}=\{x_1,\ldots,x_n\}$, entonces
\begin{equation*}
\pi_2^{-1}(U)=\bigcup_{i=1}^n V_i
\end{equation*}

, donde $\{V_j:j=1,\ldots,n\}$ son disjuntos dos a dos, $x_j\in V_j$ y $\pi_{2|V_j}:V_j\rightarrow U$ es un homeomorfismo.

Sea $W_j$ en $X$ entorno distinguido alrededor de $x_j$ para $\pi_1$, $j=1,\ldots,n$. Sea la arcocomponente $U_0$ de $\bigcap_{i=1}^n \pi_2(W_j\cap V_j)$ conteniendo a $x$. $U_0$ es abierto, pues las aplicaciones recubridoras son abiertas y como dicha intersección es un conjunto abierto, es la intersección de un conjunto de abiertos y además
\begin{equation*}
U_0\subseteq \bigcap_{i=1}^n \pi_2(W_j\cap V_j)\subseteq U
\end{equation*}

con $U$ entorno distinguido para $(X,\pi_2)$.

Por tanto, si
\begin{equation*}
U_j:=\pi_2^{-1}(U_0)\cap V_j,\qquad j=1,\ldots, n
\end{equation*}

Se tiene que $\{U_j:j=1,\ldots,n\}$ son disjuntos dos a dos, pues lo son los $V_j$, $\pi_2^{-1}(U_0)=\bigcup_{j=1}^n U_j$, y las aplicaciones $\pi_{2|U_j}:U_j \rightarrow U_0$, $j=1,\ldots,n$ homeomorfismo, pues cada $U_j$ es una restricción de $V_j$ y este era un homeomorfismo, luego la restricción de un homeomorfismo es un homeomorfismo.\\

Por definición
\begin{equation*}
U_0\subseteq \bigcap_{j=1}^n\pi_2(W_j\cap V_j)\subseteq \pi_2(W_j\cap V_j)\subseteq U
\end{equation*}

y como $\pi_{2|V_j}:V_j\rightarrow U$ es un homeomorfismo, también
\begin{equation*}
U_j=V_j\cap \pi_2^{-1}(U_0)=(\pi_{2|V_j})^{-1}(U_0)\subset W_j\cap V_j\subset W_j
\end{equation*}

de donde al ser $W_j$ entorno distinguido para $(Y,\pi_1)$, $U_j$ es entorno distinguido para $(Y,\pi_1)$, $j=1,\ldots,n$. De aquí que
\begin{equation*}
\pi_1^{-1}(U_j)=\bigcup_{p\in \pi_1^{-1}(x_j)} U_{j,p}
\end{equation*}

, donde $\{U_{j,p}: p\in \pi_1^{-1}(x_j)\}$ son disjuntos dos a dos y $\pi_{1|U_{j,p}}:U_{j,p}\rightarrow U_j$ es un homemorfismo para todo $p\in \pi_1^{-1}(x_j)$, con $j=1,\ldots,n$.\\

Como conclusión
\begin{gather*}
(\pi_2\circ \pi_1)^{-1}=\pi_1^{-1}(\pi_2^{-1}(U))=\pi_1^{-1}(\bigcup_{j=1}^n U_j)=\\
=\bigcup_{j=1}^n \pi_1^{-1}(U_j)=\bigcup_{j=1}^n(\bigcup_{p\in \pi_1^{-1}(x_j)} U_{j,p})
\end{gather*}

con $U_{j,p}$ son disjutnos dos a dos y la restricción de la composición a esto es un homeomorfismo, lo que implica que $U_0$ es un entorno distinguido.

\subsection{Ejercicio 7}
basta con tomar las siguientes aplicaciones
\begin{gather*}
h_1:\tilde{X}\rightarrow \mathbb{R}^2 \qquad (x,y,z)\mapsto (x,y)\\
f:\mathbb{R}^2\rightarrow \mathbb{S}^1\times \mathbb{R}\qquad (x,y)\mapsto (e^{2\pi xi}, y)\\
h_2:\mathbb{S}^1\times \mathbb{R}\rightarrow X \qquad (x,y)\mapsto(\sqrt{1+z^2} x,\sqrt{1+z^2}y,z)
\end{gather*}

componiendo estas tres aplicaciones tenemos una aplicación continua y sobreyectiva y que cumple perfectamente las propiedades de una aplicación recubridora.

\subsection{Ejercicio 9}
$\Leftarrow$

Tomamos $\mathcal{V}$ un recubrimiento por abiertos de $\tilde{X}$. Sea $x\in X$ y $U$ un entorno distinguido de $X$. Como $\pi$ es finita entonces $\pi^{-1}(x)=\{\tilde{x_1},\ldots,\tilde{x_n}\}$ y podemos hacer lo siguiente
\begin{equation*}
\pi^{-1}(U)\subset \bigcup_{i=1}^n V_i^x,\qquad V_i^x \cap \pi^{-1}(x)\neq \emptyset\qquad \forall i=1,\ldots,n
\end{equation*}

Al ser $\pi$ una aplicación abierta es claro que $\bigcap_{i=1}^n \pi(V_i^x)$ es abierto en $X$ contieniendo a $x$, por lo que podemos elegir $U_x\in U$ suficientemente pequeño para que
\begin{equation*}
\tilde{U}_j^x:=\pi^{-1}(U_j)\cap V_j^x
\end{equation*}
sea arcocomponente de $\pi^{-1}(U)$ para todo $j=1,\ldots,n$.\\

Como $\pi^{-1}(x)\subset \bigcup_{j=1}^n V_j^x$ y $\pi_{|\tilde{U}_j^x}:\tilde{U}_j^x\rightarrow U$ es un homeomorfismo para toda $j=1,\ldots,n$, de donde se infiere que
\begin{equation*}
\pi^{-1}(U)\subset \bigcup_{j=1}^n V_j^x \qquad U\subset \bigcap_{j=1}^n \pi(V_i^x)
\end{equation*}

Usando que $\{U:x\in X\}$ es recubrimiento por abiertos de $X$, han de existir $x_1,\ldots,x_m\in X$ tales que $X=\bigcup_{j=1}^m U_{x_j}$. Por tanto
\begin{equation*}
\tilde{X}=\pi^{-1}\left(\bigcup_{j=1}^m U_{x_j}\right) = \bigcup_{i=1}^m \left(\bigcup_{j=1}^n \tilde{U}_j^{x_i}\right)\subset \bigcup_{i=1}^m\left(\bigcup_{j=1}^n V_j^{x_i}\right)
\end{equation*}, de donde se infiere que $V_j$ con $j$ de 1 a n es un recubrimiento finito por abierto, lo que implica que $\tilde{X}$ es compacto.\\

Para ver la propiedade de Hausdorff sean $x_1,x_2\in \tilde{X}$ y llamaremos $y_1=\pi(x_1)$ y $y_2=\pi(x_2)$. Si $y_1=y_2$, basta con tomar que $x_1$ y $x_2$ estarán en arcocomponentes disjuntas (que son entornos) dentro de la preimagen del entorno distinguido de $y_1=y_2$. Y en el caso de que $y_1\neq y_2$ veamos que podemos seleccionar entornos distinguidos disjuntos, cuyas preimágenes son disjuntas y por tanto las arcocomponentes que contienen a $x_1$ y a $x_2$ son disjuntas y entornos.

$\Rightarrow$

La compacidad es trivial, pues al ser una aplicación recubridora es continua y sobreyectiva y la imagen de un compacto por una aplicación continua es compacto, lo que implica que $X$ es compacto.

Para el tema de ver que es Hausdorff basta tomar $x_1$ y $x_2$ de $X$ distintos. Tomamos entornos distinguidos de $x_1$ y $x_2$ y hacemos la preimagen de los mismos. Dentro de cada preimagen las arcocomponentes son disjuntas dos a dos, pero se puede dar que la intersección de la arcocomponente de una preimagen con la arcocomponente de otra no sea disjunta. Utilizando que $\pi$ es finita y que $\tilde{X}$ es de Hausdorff podemos elegir entornos dentro de estas arcocomponentes, de forma que estos entornos sean todos disjuntos dos a dos, de manera que al hacer la intersección de la imagen de dichos entornos nos quedan entornos en $X$ para $y_1$ e $y_2$ disjuntos, lo que implica que $X$ es Hausdorff.

\subsection{Ejercicio 14}
Para este ejercicio tomaremos $G\subset Homeo(\mathbb{T})$, con $G:=\{Id, A\}$, donde $A$ denota la aplicación antípodas. Claramente $A$ no tiene puntos fijos en $\mathbb{T}$ y por consiguiente $G$ actúa de forma propia y discontinua sobre $\mathbb{T}$. De aquí se deduce
\begin{equation*}
\pi:\mathbb{T}\rightarrow \mathbb{T}/G
\end{equation*}

es una aplicación recubridora.\\

Por otro lado, sabemos que $\mathbb{T}^+$ (la parte derecha del toro) es homeomorfa al cilidro $\mathbb{S}^1\times [0,1]$. y se pueden definir lo siguiente
\begin{equation*}
\mathbb{T}^+\xrightarrow{i} \mathbb{T}\xrightarrow{p}\mathbb{T}/G
\end{equation*}

Si denotamos a $\tilde{i}\equiv p\circ i$ esta es una identificación, entonces la identificación del cilindro es $aba^{-1}b$, es decir, la botella de Klein. Por otro lado $\mathbb{R}^2$ recubre a $\mathbb{T}$, y $\mathbb{T}$ recubre finitamente a $\mathbb{K}$, lo que implica que $\mathbb{R}^2$ recubre a $\mathbb{K}$ (ojo, esto es cierto porque $\mathbb{T}$ recubre \textbf{finitamente} a $\mathbb{K}$)

\subsection{Ejercicio 15}
$\Leftarrow$

Si $p$ es un homemomorfismo veamos que $p_*$ también lo es. Basta con considerar que $p^{-1}$ es también continua y tomaremos
\begin{gather*}
p_*\circ p^{-1}_*=(p\circ p^{-1})_*=(Id_X)_*=Id_{\Pi_1(X,x)}\\
p^{-1}_*\circ p_*=(p^{-1}\circ p)_*=(Id_{\tilde{X}})_*=Id_{\Pi_1(\tilde{X},\tilde{x})}
\end{gather*}

$\Rightarrow$

Para la vuelta nos vamos en fijar que todo espacio es recubridor de sí mismo. Sea $x\in X$ y $y_j\in \pi_j^{-1}(x)$, $j=1,2$ (en el caso de $\pi_2$ será la propia identidad de $X$). Por otro lado tenemos que
\begin{gather*}
(id_X)_*(\Pi_1(X,x))=\Pi_1(X,x)\\
(p)_*(\Pi_1(\tilde{X},y_1)=\Pi_1(X,x)
\end{gather*}

, de lo anterior se infiere que son iguales, luego se verficia que $\pi(y_1)=y_2=x$ y $p$ es un isomorfismo. 

\section{Relación 3}
\subsection{Ejercicio 1}
\begin{itemize}
\item $\mathbb{S}^2$
\begin{equation*}
aa^{-1}\Rightarrow aa^{-1}bb^{-1}\Rightarrow aa^{-1}c,c^{-1}bb^{-1}\Rightarrow \mathbb{S}^2=\left\langle a,b,c|aa^{-1}c,c^{-1}bb^{-1}\right\rangle
\end{equation*}

\item $\mathbb{T}$
\begin{equation*}
aba^{-1}b^{-1}\Rightarrow abc,\:c^{-1}a^{-1}b^{-1}\Rightarrow \mathbb{T}=\left\langle a,b,c|abc, a^{-1}b^{-1}c^{-1}\right\rangle
\end{equation*}

\item $\mathbb{K}$
\begin{equation*}
aba^{-1}b\Rightarrow abc, c^{-1}a^{-1}b\Rightarrow \mathbb{K}=\left\langle a,b,c|abc,c^{-1}a^{-1}b\right\rangle
\end{equation*}

\item $\mathbb{P}^2$
\begin{equation*}
aa\Rightarrow aabb^{-1}\Rightarrow aac, c^{-1}bb^{-1}\Rightarrow \mathbb{P}^2=\left\langle a,b,c|aac, c^{-1}bb^{-1}\right\rangle
\end{equation*}

\item $3\mathbb{P}^2$
\begin{gather*}
aabbcc\rightarrow aad, d^{-1}bbcc\Rightarrow aad, d^{-1}be, e^{-1}bcc\Rightarrow aad, d^{-1}be,e^{-1}bf, f^{-1}cc \Rightarrow\\
\Rightarrow 3\mathbb{P}^2=\left\langle a,b,c,d,e,f| aad, d^{-1}be, e^{-1}bf, f^{-1}cc\right\rangle
\end{gather*}
\end{itemize}

\subsection{Ejercicio 2}
Sabemos que $S_1\#S_2$ es orientable si y sólo si $S_1$ y $S_2$ son orientables.

\subsection{Ejercicio 3}
Por hipótesis supongamos $\chi(S)\geq -2$ entonces tenemos varias posibilidades
\begin{itemize}
\item $S$ es orientable, lo que implica que $2\geq n$, luego $n$ solo puede ser 0,1 y 2
\begin{equation*}
\left\lbrace \begin{array}{c}
\mathbb{S}^2\qquad aa^{-1}bb^{-1}cc^{-1}dd^{-1}\\
\mathbb{T}\qquad aba^{-1}b^{-1}cc^{-1}dd^{-1}\\
2\mathbb{T}\qquad aba^{-1}b^{-1}cdc^{-1}d^{-1}
\end{array}\right.
\end{equation*}

\item $S$ es no orientable, lo que implica $4\geq m$, luego $m$ solo pues ser 1,2,3 y 4
\begin{equation*}
\left\lbrace\begin{array}{c}
\mathbb{P}^2\qquad aabb^{-1}cc^{-1}dd^{-1}\\
2\mathbb{P}^2\qquad aabbcc^{-1}dd^{-1}\\
3\mathbb{P}^2\qquad aabbccdd^{-1}\\
4\mathbb{P}^2\qquad aabbccdd
\end{array}\right.
\end{equation*}
\end{itemize}

Por otro lado si tenemos que es un octógono se tiene que $\chi(S)=1-4+V=-3+V$ como obligatoriamente tiene que haber al menos un vértice entonces es claro que $\chi(S)\geq -2$

\subsection{Ejercicio 4}
Debemos estudiar varios casos
\begin{itemize}
\item La superficie compacta es orientable
\begin{equation*}
\chi(S)=-4=2-2n\Rightarrow n=3\Rightarrow S\cong 3\mathbb{T}
\end{equation*}

\item La superficie compacta es no orientable
\begin{equation*}
\chi(S)=-4=2-n\Rightarrow n=6\Rightarrow S\cong 6\mathbb{P}^2
\end{equation*}
\end{itemize}

\subsection{Ejercicio 5}
\begin{itemize}
\item $a_1a_2\ldots a_na_1^{-1}a_2^{-1}\ldots a_n^{-1}$

Esta superficie es claramente orientable y únicamente debemos diferencias si $n$ es par o impar. Si es par hay un sólo vértice, mientras que si es impar hay dos.

En el primer caso $S\cong \frac{n}{2}\mathbb{T}$, mientras que en el segundo $S\cong \frac{n-1}{2}\mathbb{T}$

\item $a_1a_2\ldots a_na_1^{-1}a_2^{-1}\ldots a_{n-1}^{-1}a_n$

Aquí no hay que diferencias entre par o impar, $S\cong n\mathbb{P}^2$

\item $a_1^{-1}a_2\ldots a_na_1^{-1}a_2^{-1}\ldots a_n^{-1}$

De nuevo no hay diferencias entre par o impar, $S\cong n\mathbb{P}^2$

\item $a_1^{-1}a_2\ldots a_na_1^{-1}a_2^{-1}\ldots a_{n-1}^{-1}a_n$

$S\cong (n-1)\mathbb{P}^2$
\end{itemize}

\subsection{Ejercicio 6}
Como debe ser la suma conexa igual a $4\mathbb{P}^2$ es una superficie no orientable y $\chi(4\mathbb{P}^2)=-2$\\

\begin{itemize}
\item Solo una de las dos superficies es no orientable
\begin{equation*}
\chi(S_1\#S_2)=2-n+2-2m-2=-2\Rightarrow 2-n-2m=-2\Rightarrow 4=n+2m
\end{equation*}

De aquí se deduce que
\begin{equation*}
\left\lbrace\begin{array}{c}
n=0\:\wedge m=2 \Rightarrow S_1\cong \mathbb{S}^2\:\wedge S_2\cong 2\mathbb{T}\\
n=2\:\wedge m=1 \Rightarrow S_1\cong 2\mathbb{P}^2\:\wedge S_2\cong \mathbb{T}
\end{array}\right.
\end{equation*}

\item Ambas superficies son no orientables
\begin{equation*}
\chi(S_1\#S_2)=2-n+2-m-2=-2\Rightarrow 2-n-2m=-2\Rightarrow 4=n+m
\end{equation*}
De aquí se deduce

\begin{equation*}
\left\lbrace\begin{array}{c}
n=1\:\wedge m=3 \Rightarrow S_1\cong \mathbb{P}^2\:\wedge S_2\cong 3\mathbb{P}^2\\
n=2\:\wedge m=2 \Rightarrow S_1\cong 2\mathbb{P}^2\:\wedge S_2\cong 2\mathbb{P}^2
\end{array}\right.
\end{equation*}

\end{itemize}

\subsection{Ejercicio 7}
En este caso $4\mathbb{T}$ es una superficie orientable, luego tanto $S_1$ como $S_2$ deben de ser orientables. Por otro lado $\chi(4\mathbb{T})=2-2*4=-6$
\begin{equation*}
\chi(S_1\#S_2)=\chi(S_1)+\chi(S_2)-2=2-2n+2-2m-2=2-2*(n+m)=-6\Rightarrow (n+m)=4
\end{equation*}

Luego las únicas posibilidades son
\begin{equation*}
\left\lbrace\begin{array}{c}
n=0\:\wedge\:=4\quad S_1\cong \mathbb{S}^2\:\wedge\:S_1\cong 4\mathbb{T}\\
n=1\:\wedge\:=3\quad S_1\cong \mathbb{T}\:\wedge\:S_1\cong 3\mathbb{T}\\
n=2\:\wedge\:=2\quad S_1\cong 2\mathbb{T}\:\wedge\:S_1\cong 2\mathbb{T}
\end{array}\right.
\end{equation*}

\subsection{Ejercicio 8}
Volvemos al caso del ejercicio 6 y vamos probando uno a uno (en este caso $\chi(8\mathbb{P}^2)=2-8=-6$
\begin{itemize}
\item Una de las superficies es orientable y la otra no

$\chi(S_1\#S_2)=2-n+2-2m-2=-6\Rightarrow 8=n+2m$

Entonces las posibilidades que nos quedan son
\begin{equation*}
\left\lbrace\begin{array}{c}
n=2\:\wedge\:m=3\qquad S_1\cong 2\mathbb{P}^2\:\wedge\:S_2\cong 3\mathbb{T}\\
n=4\:\wedge\:m=2\qquad S_1\cong 4\mathbb{P}^2\:\wedge\:S_2\cong 2\mathbb{T}\\
n=6\:\wedge\:m=1\qquad S_1\cong 6\mathbb{P}^2\:\wedge\:S_2\cong \mathbb{T}\\
n=8\:\wedge\:m=0\qquad S_1\cong 8\mathbb{P}^2\:\wedge\:S_2\cong \mathbb{S}^2
\end{array}\right.
\end{equation*}

\item Ambas superficies son no orientables, lo que nos indica que

$\chi(S_1\#S_2)=2-n+2-m-2=-6\Rightarrow 8=n+m$

Entonces nos quedan las siguientes posibilidades
\begin{equation*}
\left\lbrace\begin{array}{c}
n=1\:\wedge m=7\qquad S_1\cong \mathbb{P}^2\:\wedge\:S_2\cong 7\mathbb{P}^2\\
n=2\:\wedge m=6\qquad S_1\cong 2\mathbb{P}^2\:\wedge\:S_2\cong 6\mathbb{P}^2\\
n=3\:\wedge m=5\qquad S_1\cong 3\mathbb{P}^2\:\wedge\:S_2\cong 5\mathbb{P}^2\\
n=4\:\wedge m=4\qquad S_1\cong 4\mathbb{P}^2\:\wedge\:S_2\cong 4\mathbb{P}^2
\end{array}\right.
\end{equation*}

y todas las obtenidas de intercambiar los papeles entre $S_1$ y $S_2$.
\end{itemize}

\subsection{Ejercicio 9}
Por un lado al hacer la función característica de Euler de $S_1$, que es no orientable, nos queda que
\begin{equation*}
\chi(S_1)=1-3+1=-1\Rightarrow S_1\cong 3\mathbb{P}^2
\end{equation*}

Por su parte $S_2$ es orientable y nos queda
\begin{equation*}
\chi(S_2)=1-3+2=0\Rightarrow S_2\cong \mathbb{T}
\end{equation*}

Y por último se obtiene que $S_1\#S_2$ es no orientable y $\chi(S_1\#S_2)=-1+0-2=-3\Rightarrow S_1\#S_2\cong 5\mathbb{P}^2$

\subsection{Ejercicio 10}
De nuevo $S_1$ es no orientable y obtenemos que
\begin{equation*}
\chi(S_1)=1-4+2=-1\Rightarrow S_1\cong 3\mathbb{P}^2
\end{equation*}

Por otra parte $S_2$ es también no orientable de donde se obtiene que
\begin{equation*}
\chi(S_2)=1-3+1=-1\Rightarrow S_2\cong 3\mathbb{P}^2
\end{equation*}

Y claramente se obtiene que la suma conexa es $S_1\#S_2\cong 6\mathbb{P}^2$

\subsection{Ejercicio 11}
\begin{gather*}
abacb^{-1}c^{-1}\Rightarrow^{Corta} abd, d^{-1}acb^{-1}c^{-1}\\
\Rightarrow^{reflejar}abd, cbc^{-1}a^{-1}d\\
\Rightarrow^{rotar}bda, a^{-1}dcbc^{-1}\\
\Rightarrow^{pegar}bddcbc^{-1} \\
\Rightarrow^{cortar}bddce, e^{-1}bc^{-1}\\
\Rightarrow^{reflejar}bddce, cb^{-1}e\\
\Rightarrow^{rotar}ddceb,b^{-1}ec\\
\Rightarrow^{pegar}ddceec\\
\Rightarrow^{cortar}ddcf, eecf^{-1}\\
\Rightarrow^{reflejar}ddcf,fc^{-1}e^{-1}e^{-1}\\
\Rightarrow^{rotar}fddc,c^{-1}e^{-1}e^{-1}f\\
\Rightarrow^{pegar}fdde^{-1}e^{-1}f\\
\Rightarrow^{rotar}dde^{-1}e^{-1}ff\\
\Rightarrow^{renombrar}ddeeff
\end{gather*}

\subsection{Ejercicio 12}
Teneiendo en cuenta que 
\begin{equation*}
\mathcal{A}(\Pi_1(S_n))\cong \mathbb{Z}^{2n}\qquad \mathcal{A}(\Pi_1(S^*_{n+1}))\cong\mathbb{Z}_2\times \mathbb{Z}^n
\end{equation*}
, donde $S_n$ es el espacio asociado al esquema binario $\prod_{j=1}^n a_jc_ja_j^{-1}c_j^{-1}$, mientras que $S^*_n$ es el espacio asociado al esquema binario $\prod_{j=1}^n a_ja_j$.

Po lo que nos dice el enunciado el único esquema binario factible es $aabbccddeeffjjhh$, el cual es claramente $8\mathbb{P}^2$. Por otro lado la característica de Euler de este es $\chi(8\mathbb{P}^2)=2-8=-6$ que es claramente menor que 7, con lo que hemos obtenido la única superficie compacta que verifica lo pedido.
\end{document}